
\section{جنبه‌های الگوریتمی}\label{algorithm_aspect}

\subsection{ تولید پارامترها}\label{parameter_generate}

 اعداد اولی که در طرح خود استفاده می‌کنیم به فرم 
$p = \ell_A^{e_A} \ell_B^{e_B} \cdot f \pm 1$
می باشند. دلیل این امر آن است که برای تامین امنیت طرح خود لازم است تا ابتدا اعداد اول ثابت  
$\ell_A$
و
$\ell_B$
را به‌صورت مجزا از عدد اول
$p$
، با ویژگی 
$\ell_A^{e_A} \approx \ell_B^{e_B}$ 
(به این معنی که از نظر بیتی هم اندازه هستند)
انتخاب کنیم. مطمئنا با ضرب مقادیر 
$\ell_A^{e_A}$
و
$\ell_B^{e_B}$
عدد اولی خاصل نخواهد شد، در بهترین حالت با اضافه یا کم کردن مقدار یک به این حاصلضرب می‌توان به یک عدد اول رسید و اگر نتیجه حاصل نشد می‌توانیم مقدار یک را با مضربی از 
$ \ell_A^{e_A} \ell_B^{e_B} $
جمع یا تفریق کنیم. این مضرب در فرم بالا همان مقدار متغیر
$f$
می‌باشد. بنابراین عدد اول استفاده شده در طرح ما به‌یکی از فرم‌های زیر خواهد بود: 
\begin{center}
	$p = \ell_A^{e_A} \ell_B^{e_B} \cdot f - 1$~
	\text{یا}~
	$p = \ell_A^{e_A} \ell_B^{e_B} \cdot f + 1$
\end{center}

بروکر در
\cite{broker}
نشان داده است برای هر عدد اول 
$p = \ell_A^{e_A} \ell_B^{e_B} \cdot f \pm 1$
، می‌توان به راحتی یک خم بیضوی سوپرسینگولار 
$E$
روی میدان
$\mathbb{F}_{p^2}$
با مرتبه 
$({p \mp 1})^2 = (\ell_A^{e_A} \ell_B^{e_B} \cdot f)^2$
به دست آورد. دلیل این امر هم آن است که اگر
$E$
یک خم بیضوی روی میدان 
$\mathbb{F}_p$
باشد، آنگاه 
$\mathbb{F}_p$
-نقاط روی خم به شکل زیر می‌باشد:
$$ E(\mathbb{F}_p) = \{ (x,y) \in \mathbb{F}_p \times \mathbb{F}_p ~ | ~ y^2 = x^3+Ax+B \} \cup \{ \infty \}$$
که نتیجه می‌شود که :
$$ E(\mathbb{F}_p) \subseteq (\mathbb{F}_p \times \mathbb{F}_p) \cup \{ \infty \} $$
و چون مجموعه‌ی سمت راست متناهی( از مرتبه‌ی 
$p^2 + 1 $)
 است لذا مجموعه‌ی سمت چپ یعنی 
$E(\mathbb{F}_p)$
نیز متناهی است. بنابراین :
$$ \# E(\mathbb{F}_p) \leq p^2+1 $$
حال اگر یک تغییر کوچک به معادله‌ی بالا اعمال کنیم، خواهیم داشت:
$$ \# E(\mathbb{F}_p) - 1 \leq p^2 $$
حال، اگر 
$$
p = \ell_A^{e_A} \ell_B^{e_B} \cdot f + 1
~~ \text{یا} ~~
p = \ell_A^{e_A} \ell_B^{e_B} \cdot f - 1
$$
آنگاه
$$
p^2 = (\ell_A^{e_A} \ell_B^{e_B} \cdot f)^2 + 2 (\ell_A^{e_A} \ell_B^{e_B} \cdot f) + 1
~~ \text{یا} ~~
p^2 = (\ell_A^{e_A} \ell_B^{e_B} \cdot f)^2 - 2 (\ell_A^{e_A} \ell_B^{e_B} \cdot f) - 1
$$
که به‌طور خلاصه خواهیم داشت:
$$ p^2 \mp 1 = (\ell_A^{e_A} \ell_B^{e_B} \cdot f )^2 \pm 2 \ell_A^{e_A} \ell_B^{e_B} \cdot f   $$
و از آنجا که 
$ (\ell_A^{e_A} \ell_B^{e_B} \cdot f) = p $
، بنابراین فرم بالا به‌صورت زیر خلاصه می‌شود:
$$ p^2 \mp 2p \mp 1 = (\ell_A^{e_A} \ell_B^{e_B} \cdot f)^2 $$
که اگر دوباره آن‌ را خلاصه کنیم، خواهیم داشت:
$$ (p \mp 1)^2 = (\ell_A^{e_A} \ell_B^{e_B} \cdot f)^2 $$
که همان مرتبه‌ی گفته شده با روش بروکر می‌باشد.

\subsection{پیدا کردن مولدهای زیرگروه تابی}\label{find_generators_of_torsion_points}

برای انتخاب نقاط مولد زیرگروه
$E_0[\ell_A^{e_A}]$
، می‌توان یک نقطه تصادفی 
$P \:  {\in}_R \: E_0(\mathbb{F}_{p^2}) $
انتخاب و آن را در 
${(\ell_B^{e_B} \cdot f )}^2$
ضرب کرد تا نقطه
$P'$
با مرتبه توانی از 
% $\ell_A^{e_A}$
$\ell_A$
حاصل شود. از آنجا که عامل‌های عدد اول
$p$،
$\ell_A^{e_A}$
و
$\ell_B^{e_B}$
می‌باشند، احتمالا 
$P'$
از مرتبه
$\ell_A^{e_A}$
خواهد بود؛ برای اثبات این ادعا می‌توان با ضرب 
$P'$
در توان‌هایی از 
$\ell_A$
آن را بررسی کرد. اگر بررسی موفقیت آمیز بود آنگاه 
$P_A = P'$
در نظر می‌گیریم در غیر اینصورت به دنبال یافتن نقطه‌ای دیگر برای
$P$
می‌شویم. برای به دست آوردن نقطه دوم مولد یعنی
$Q_A$
از مرتبه‌ی
$\ell_A$
، می‌توان از همین روش استفاده کرد.برای بررسی این که آیا نقطه
$Q_A$
از نقطه
$P_A$
متفاوت است ، می‌توان به راحتی با استفاده از زوجیت وایل و محاسبه
$e(P_A,Q_A)$
در میدان
$E[\ell_A]$
، بررسی کرد که آیا نتیجه از مرتبه 
$\ell_A$
می‌باشد یا خیر ؛ 
برای اطمینان از اینکه نقطه‌ي
$Q_A$
متفاوت از نقطه‌ی 
$P_A$
می‌باشد می‌توانیم از گزاره زیر استفاده کنیم:
\proposition{

اگر
$P_A , Q_A \in E[ \ell_A ]$
 و 
$ \ell_A $
 عددی اول باشد آنگاه 
$$ 
e_n(P_A,Q_A) = 1
~~ \text{اگر و تنها اگر} ~~
Q_A = k P_A 
$$
}

\refproof{~
\\
\begin{enumerate}
\item{
اگر فرض شود به ازای یک 
$k$
،
$Q_A = kP_A$
در این‌صورت:
$$ e_n(P_A,Q_A) = e_n(P_A,kP_A) = e_n(P_A,P_A)^k = 1^k = 1 $$
}

\item {
همچنین اگر 
$e_n(P_A,Q_A) = 1$
، در اینصورت
$R \in E[\ell_A]$
را چنان اختیار می‌کنیم که
$E[\ell_A] = \langle P,R \rangle$
، بنابراین 
$\partial = e_n(P_A,R)$
یک ریشه 
n
-ام اولیه واحد است. پس:
$$
Q_A \in E[n] = \langle P_A,R \rangle \longrightarrow \exists 0 \leqslant k,l \leqslant \ell_A - 1 ~~, ~~~~
Q_A = kP_A + \ell R
$$
اکنون :
$$ 1 = e_n(P_A,Q_A) = e_n(P,kP + \ell R) = e_n(P,P)^kr_n(P,R)^{\ell} = \partial^{\ell} $$
بنابراین
$ \partial ^ {\ell} = 1$
و درنتیجه
$\ell = 0$
، پس 
$Q_A = k P_A$.
}
\end{enumerate}~
} % end of refproof
\\
\textbf{توجه .}
انتخاب نقاط مولد ، هیچ گونه تاثیری روی امنیت این طرح ندارد ؛ از آنجا که هر کدام از نقاط مولد با استفاده از لگاریتم گسسته توسیع یافته، قابل تبدیل به یکدیگر می‌باشند. چنانچه در 
\cite{teske}
اشاره شده است این محاسبه به راحتی در زیرگروه
$E[\ell_A]$
قابل انجام می‌باشد. 
\subsection{تبادل کلید }\label{key_exchange}
یک از مهمترین ارکان هر سیستم رمزنگاری مربوط به تبادل کلید می‌باشد. به‌عبارت دیگر زمانی‌که یک سیستم رمزنگاری معرفی می‌شود در ابتدا بررسی می‌شود که آیا می‌توان پروتکل تبادل کلید را پیاده‌سازی کرد یا خیر. بعد از معرفی سیستم رمزنگاری همسانی-مبنا، آقای جائو در 
\cite{}
 به معرفی تبادل کلید در همسانی‌ها پرداخت. از آنجا که این موضوع در ادامه بحث مورد استفاده قرار می‌گیرد، این پروتکل رو تشریح می‌کنیم.
\\
 
پروتکل تبادل کلید نوعی از پروتکل دیفای-هلمن است که طبق شکل ۱ صورت می‌پذیرد. ایده‌ی کلی این پروتکل آن است که شخصی مانند آرش، همسانی 
$\phi$
و شخص دیگری همچون بابک همسانی 
$\psi$
را به عنوان کلیدخصوصی انتخاب می‌کنند و با یک خم سوپرسینگولار عمومی همچون
$E_0$
بااستفاده از پروتکل قدم‌زدن روی گراف به تولید یک خم خصوصی همچون
$E_A$
و
$E_B$
می‌پردازند که بیانگر کلیدخصوصی آنها همچون دیگر پروتکل‌های رمزنگاری می‌باشد. در ادامه با انجام محاسباتی که در ادامه تشریح خواهد شد به‌طور مجزا به تولید یک خم سوپرسینگولار همچون 
$E_{AB}$
خواهند پرداخت که به عنوان کلیدعمومی‌شان درنظر گرفته می‌شود.
\\
\begin{enumerate}
\item{
	در ابتدا، یک خم سوپرسینگولار دلخواه در میدان
	$\mathbb{F}_{p^2}$
	همچون
	$E_0$ 
	و دو جفت نقاط 
	$\{P_A,Q_A\}$
	و
	$\{P_B,P_B\}$
	که مولد زیرگروه‌های تابی
	$E_[\ell_A^{e_A}]$
	و
	$E_[\ell_B^{e_B}]$
	می‌باشند را به‌عنوان پارامترهای عمومی پروتکل درنظر می‌گیریم.
	
}
\item{
\begin{enumerate}
\item{
درادامه آرش دو عنصر تصادفی همچون
$m_A,n_A \in \mathbb{Z}/ \ell_A^{e_A} \mathbb{Z}$
(که هردو همزمان به 
$\ell_A$
بخش‌پذیر نیستند) را انتخاب می‌کند و همسانی
$\phi_{A} : E_0 \rightarrow E_A$
را که هسته‌ی آن
$K_A := \langle [m_A]P_A + [n_A]Q_A$
می‌باشد را محاسبه می‌کند.درادامه آلیس 
$\{ \phi_A(P_B) , \phi_A(Q_B) \} \subset E_A$
را نیز محاسبه می‌کند.
}
\item{
به‌طورمشابه، بابک نیز دو عنصر تصادفی همچون
$m_B,n_B \in \mathbb{Z}/ \ell_B^{e_B} \mathbb{Z}$
را انتخاب و همسانی
$\phi_{B} : E_0 \rightarrow E_B$
با هسته‌ی
$K_B := \langle [m_B]P_B + [n_B]Q_B$
را به‌همراه نقاط
$\{ \phi_B(P_A) , \phi_B(Q_A) \}$
محاسبه می‌کند.

}
\end{enumerate}

\item {
\begin{enumerate}
\item{
در ادامه آرش پس از دریافت 
$E_B$
و
$\phi_B(P_A) , \phi_B(Q_A) \in E_B$
از جانب بابک، به محاسبه‌ی همسانی
${\phi}'_A : E_B \rightarrow E_{AB}$
با هسته‌ی
$\langle [m_A]\phi_{B}(P_A) + [n_A]\phi_{B}(Q_A) \rangle$
می‌پردازد.

}
\item{
  بابک نیز با دریافت
$E_A$
 به محاسبه‌ی همسانی
${\phi}'_B : E_A \rightarrow E_{AB}$
با هسته‌ی
$\langle [m_B]\phi_{A}(P_B) + [n_B]\phi_{A}(Q_B) \rangle$
می‌پردازد.
}
\end{enumerate}
}

}

\item{
آرش و بابک برای دسترسی به یک کلیدخصوصی مشترک می‌توانند
$j$
-پایای خم زیر را محاسبه کنند:
$$
E_{AB} = {\phi}_B'(\phi_A(E_0)) = {\phi}_A'(\phi_B(E_0)) = 
E_0 / \langle [m_A]P_A + [n_A]Q_A , [m_B]P_B + [n_B]Q_B  \rangle
$$

}
\end{enumerate}





\subsection{ساده‌سازی نقاط تاب‌دار}\label{sampling_torsion_points}
% ===========================================================================================
% Algorithm 1  three-point ladder to compute P + [t]Q
% ===========================================================================================
\begin{algorithm}\label{algorithm_ladder}
	\caption{نردبان سه-نقطه‌ای برای محاسبه‌ی $P + [t]Q$}
	\begin{latin}
		%\resetlatinfont
		\textbf{Input :} ~ t,P,Q
		\begin{algorithmic}[1]
			\State Set $A=0, B=Q, C=P$
			\State Compute $Q-P ;$
			\For{ $ i= |t| \ \textbf{to} \ 1 $ }
			\State Let $t_i$ be the $i$-th bit of t;
			\If {$t_i = 0 $} 
			\State $A=2A, B=dadd(A,B,Q), C=dadd(A,C,P);$
			\Else
			\State $A=dadd(A,B,Q), B=2B, C=dadd(B,C,Q-P);$
			\EndIf
			\EndFor	
		\end{algorithmic}
		\textbf{Output :} ~ $C=P+[t]Q$
	\end{latin}
\end{algorithm}~
% ===========================================================================================
% ===========================================================================================
در طرح خود از نقاط تصادفی متنوعی با مرتبه‌ی خاص همچون 
$\ell_A^{e_A}$
و
$\ell_B^{e_B}$
به‌مراتب برای ساخت یک زیرگروه استفاده می‌کنیم. از آنجا که نقاط موردنظر ما از زیرگروه‌های تابی  
$E[\ell_A^{e_A}]$
و
$E[\ell_B^{e_B}]$
با دو مولد
$P$
و
$Q$
ساخته می‌شوند لذا این نقاط به فرم
$\langle [m]P + [n] Q \rangle $
خواهند بود. البته
$m$
و
$n$
همزمان نباید با
$\ell^e$
موردنظر بخش‌پذیر باشند. بنابراین نتیجه می‌گیریم که که اگر روشی ارائه دهیم تا ساخت این زیرگروه‌ها با محاسبات کمتری انجام پذیرند، به‌طور کلی طرح ما بهینه‌تر خواهد شد.
\\
می‌توان فرض کرد که هر
$m$
، دارای عنصر وارون در پیمانه‌ی مرتبه‌ی گروه می‌باشد( این فرض هیچ خدشه‌ای به زیرگروه وارد نمی‌کند). در این حالت 
$R' = P + [m^{-1}n]Q$
زیرگروهی همانند دیگر مولدها خواهد بود. محاسبه 
$R'$
با روش استاندارد  
\textbf{دوبرابر-و-جمع} 
\LTRfootnote{double-and-add}
نیاز به نصف عملیات محاسبات
$[m]P + [n]Q$
معمولی را دارا می‌باشد( برای روش های بهتر محاسبه عملیات معمولی به مراجعه 
\cite{ antipa, elgamal, solinas}
شود). با این حال،  محاسبه 
$P + [m^{-1}n]Q$
با روش دوبرابر-و-جمع، یک حفره امنیتی (اشکال بزرگ) را داراست : در برابر حملات 
\textbf{آنالیز قدرت ساده } 
یا
\textbf{SPA}
\cite{spa}
آسیب پذیر می‌باشد. برای جلوگیری از این حمله می‌توان از روش
\textbf{نردبان مونت‌گومری}
\LTRfootnote{Montgomery ladder}
\cite{montgomery}
برای محاسبه 
$[m^{-1}n]Q$ 
استفاده کرد و سپس 
$P$
را به آن اضافه کرد، اما این روش به طور قابل ‌توجهی کند می‌باشد.
به‌منظور رفع دو مشکل کندی و حمله‌ی
\text{SPA}
، الگوریتم
\ref{algorithm_ladder}
 را ارائه می‌دهیم که بیانگر یک روش بسیار موثرتری  می‌باشد و مستقیما 
$P + [m^{-1}n]Q$
را محاسبه می‌کند. ایده اصلی این طرح ساده است : در هر تکرار ، ثبات های 
$A$
و
$B$
و
$C$
محتوی مقدارهای به ترتیب
$[x]Q$
و
$[x+1]Q$
و
$P+[x]Q$
می ‌باشند ، که 
$x$
حاوی ارزش چپ ترین بیت
$m^{-1}n$
می‌باشد.تابع 
$dadd(A,B,C)$
مورد استفاده شده در الگوریتم نیز معرف جمع تفاضلی 
\LTRfootnote{differential addition}
\cite{montgomery}
می‌باشد. 
\\
پیاده سازی جمع تفاضلی در خم‌های مونت‌گومری به کارآمدی روش دوبرابر-و-جمع ساده روی خم های دوقولوی ادوارد
\LTRfootnote{twisted Edwards curves}
\ref{model_choice}
می‌باشد، بنابراین درادامه به معرفی خم‌های مونت‌گومری خواهیم پرداخت.

\subsection{\bf   محاسبه همسانی‌های با درجه هموار}\label{smooth_isogeny}
\LTRfootnote{Computing smooth degree isogenies}

محاسبه‌ی همسانی یکی از پرهزینه‌ترین محاسبات در سیستم‌های همسانی‌-مبنا می‌باشند. از آنجا که در طرح خود نیز به‌مراتب به محاسبه‌ی همسانی‌ها با درجه‌ی معینی می‌پردازیم بنابراین  لازم است تا روشی سریع برای این امر معرفی کنیم. البته این روش می‌تواند در تمامی طرح‌ها مورداستفاده قرار گیرد، به‌عنوان مثال می‌توان در محاسبات همسانی در پروتکل تبادل کلیدی که در بخش قبل بین آرش و بابک انجام می‌پذیرد اشاره کرد.
\\
 فرض کنیم 
$E$
یک خم بیضوی و 
$R$
یک نقطه از مرتبه 
$\ell^e$
باشد. هدف ما محاسبه تصویر خم 
$E/ \langle R \rangle $
و ارزیابی همسانی 
$\phi : E \rightarrow E/ \langle R \rangle $
در بعضی نقاط روی خم 
$E$
می‌باشد. 
% ==============================================================================================
\begin{figure}[H]\label{}
	\begin{center}
		
		
		\caption{
			ساختمان محاسبات ساخت 
			$\phi = {\phi}_5 \circ \cdots  \circ {\phi}_0$    
		}
		
	\end{center}
\end{figure}
% ==============================================================================================
اگر درجه‌ی نگاشت 
$\phi$
هموار باشد، می‌توان آن را به زنجیره‌ای از
$\ell$
-همسانی ها تجزیه کرد. اگر
$E_0 = E$
و
$R_0 = R$
در نظر بگیریم ، آنگاه برای هر 
$0 \leq i < e $
می‌توان مقادیر زیر را در نظر گرفت :
$$
E_{i+1} = E_i / \langle \ell^{e-i-1}R_i \rangle , \quad
\phi_i : E_i \rightarrow E_{i+1} , \quad
R_{i+1} = \phi_i(R_i).
$$
چنانکه 
$E / \langle R \rangle = E_e$
و
$\phi = \phi_{e-1} \circ \cdots \circ \phi_0$
می‌باشد.

توجه به این نکته لازم است که از آنجا که زیرگروه 
$\ell$
-تابی 
$\langle R_i \rangle $
خم 
$E_i$
مشخص می‌باشند ، 

خم بیضوی 
$E_{i+1}$
و همسانی
$\phi_i$
می‌توانند توسط فرمول ولو 
\LTRfootnote{Velu's formulas}
\cite{velu}
به راحتی محاسبه شوند.
در 
\cite{jao2011towards}
، دو پیشنهاد برای داشتنی پیچیدگی درجه دو برای 
$e$
بیان شده است  ؟؟؟؟؟.
% به هر حال می‌توانیم این کار را بهتر انجام دهیم. 
\\
شکل بالا خلاصه‌ای از ساختار محاسباتی مسئله برای 
$e = 6$
می‌باشد. نقطه‌های توپر این گراف نشان دهنده نقاط می‌باشد. نقطه‌های موجود در یک سطح افقی نشان دهنده آن است که این نقاط از یک مرتبه می‌باشند و همچنین نقطه‌های روی خط مورب چپین نشان دهنده آن است که این نقطه‌ها همگی متعلق به یک خم می‌باشند. یال‌های نقطه‌چین همگی جهت‌دار و به سمت پایین می‌باشند ؛ یال‌های چپین معرف آن هستند که نقطه‌ها 
$\ell$
برابر شده‌اند و یال‌های راست‌چین هم یک 
$\ell$
ـ همسانی را نشان می‌دهند.
% try to better writing ..
در ابتدای اجرای الگوریتم ، تنها نقطه 
$R_0$
را در اختیار داریم. به بیان دیگر هدف ما در این الگوریتم  محاسبه تمام نقاط روی خط پایانی توسط نقطه آغازین 
$R_0$
می‌باشد (ورودی این الگوریتم نقطه 
$R_0$
و خروجی این الگوریتم نقاط 
$[\ell^5]R_0$
،
$[\ell^4]R_1$
،
$[\ell^3]R_2$
،
$[\ell^2]R_3$
،
$[\ell^1]R_4$
و
$R_5$
می‌باشد).
در واقع با دانستن نقطه 
$[\ell^{e-i-1}]R_i$
، می‌توانیم هسته همسانی
$\phi_i$
را به تعداد
$\mathcal{O}(\ell)$
جمع نقاط ، محاسبه کنیم ؛ که در این صورت پیچیدگی محاسبات به طور قابل توجهی کم می‌شود. در ادامه می‌توانیم از طریق فرمول ولو ، همسانی 
$\phi_i$
و خم 
$E_{i+1}$
را محاسبه کنیم. 
% complete next statemnet correctly !!!
\\
برای فهم بیشتر این الگوریتم مراحل ذکر شده در مثال 
$e=6$
را مرحله به مرحله نمایش می‌دهیم :
\begin{itemize}
	
	\item[] {}
	\begin{flushleft}
		$
		i=0 \Rightarrow \quad 
		E_1 = E_0 / \langle \ell^{4}R_0 \rangle , \quad 
		\phi_0 : E_0 \rightarrow E_1 , \quad 
		R_1 = \phi_0(R_0)
		$
	\end{flushleft}
	
	\item[] {}
	\begin{flushleft}
		$
		i=1 \Rightarrow \quad 
		E_2 = E_1 / \langle \ell^{3}R_1 \rangle , \quad 
		\phi_1 : E_1 \rightarrow E_2 , \quad 	R_2 = \phi_1(R_1) = \phi_1(\phi_0(R_0))
		$
	\end{flushleft}
	
	\item[] {}
	\begin{flushleft}
		$
		i=2 \Rightarrow \quad 
		E_3 = E_2 / \langle \ell^{2}R_2 \rangle , \quad 
		\phi_2 : E_2 \rightarrow E_3 , \quad 	R_3 = \phi_2(R_2) = \phi_2(\phi_1(\phi_0(R_0)))
		$
	\end{flushleft}
	
	\item[] {}
	\begin{flushleft}
		$
		i=3 \Rightarrow \quad 
		E_4 = E_3 / \langle \ell^{1}R_3 \rangle , \quad 
		\phi_3 : E_3 \rightarrow E_4 , \quad 	R_4 = \phi_3(R_3) = \phi_3(\phi_2(\phi_1(\phi_0(R_0))))
		$
	\end{flushleft}
	
	\item[] {}
	\begin{flushleft}
		$
		i=4 \Rightarrow \quad 
		E_5 = E_4 / \langle \ell R_4 \rangle , \quad 
		\phi_4 : E_4 \rightarrow E_5 , \quad 	R_5 = \phi_4(R_4) = 
		\phi_4(\phi_3(\phi_2(\phi_1(\phi_0(R_0)))))
		$
	\end{flushleft}
	
	\item[] {}
	\begin{flushleft}
		$
		i=5 \Rightarrow \quad 
		E_6 = E_5 / \langle R_5 \rangle , \quad 
		\phi_5 : E_5 \rightarrow E_6 , \quad 	R_6 = \phi_5(R_5) = 
		\phi_5(\phi_4(\phi_3(\phi_2(\phi_1(\phi_0(R_0))))))
		$
	\end{flushleft}
	
\end{itemize}


\subsection{\bf انتخاب مدل}\label{model_choice}

بعد از ارائه‌ی یک طرح لازم است تا آن طرح بهینه‌سازی شود. به‌عبارت‌دیگر زمانی که یک طرح ارائه و تایید شد برای پیاده‌سازی آن باید سرعت اجرای طرح را تا جای ممکن افزایش داد. به‌طورمثال می‌توانیم تعداد عملیاتی که در یک الگوریتم انجام می‌شود را به حداقل رساند یا عملیات‌های سنگین تر را به عملیات های با بار پیچیدگی کمتر از لحاظ زمان اجرا جایگزین کرد.
معمولا در  زمان محاسبه پیچیدگی یک الگوریتم از عملیات تفریق، جمع و مقایسه صرف‌نظر می‌کنند. دلیل این امر آن است که  این عملبات ، پیچیدگی محاسبات بالایی ندارند و به‌عنوان عمل‌های پایه‌ در  هر الگوریتم فرض می شوند. از جمله اعمالی که در سرعت اجرای یک الگوریتم می‌تواند تاثیرگذار باشد می‌توان به  عملیات وارون، ضرب و توان اشاره  کرد.  برای بهینه‌سازی یک الگوریتم تلاش می‌شود که این عملیات ‌ها به حداقل برسند. در ادامه  از علائم
$I$
،
$M$
و
$S$
به‌منظور عملیات وارون، ضرب و توان استفاده می‌کنیم. 
از جمله عملیات سنگین و زمان‌بری که در طرح امضای خود می‌توانیم نام ببریم عملیات 
\textbf{دوبرابرکردن}
، 
\textbf{جمع}
، 
\textbf{محاسبه}
 و 
\textbf{ارزیابی}
 همسانی‌ها خواهد بود.
چنانچه ذکر شد پس از ارائه‌ی طرح خود مصمم هستیم تا سرعت اجرای الگوریتم ها در طرح خود را افزایش دهیم. یکی از راه حل‌های ممکن این است که به‌جای استفاده از یک خم بیضوی معمولی با معادله‌ی وایرشتراس، از خم بیضوی مونت‌گومری استفاده کنیم. مزیت استفاده از خم مونت‌گومری را پس از تعریف آن ذکر خواهیم کرد.
\\
\definition
یک خم مونت‌گومری در میدان 
$\mathbb{F}_q$
یک خم بیضوی به فرم زیر می‌باشد:
$$ M_{B,A} : By^2 = x^3 + Ax^2 + x $$
در این خم برای نقاط
$P = (x_P,y_P)$
و
$Q = (x_Q, y_Q)$
، نقطه‌‌ی
$R = P+Q = (x_R,y_R)$
به‌صورت زیر محاسبه می‌شود:
\[
\begin{gathered}
x_R = B{\lambda}^2 - (x_P + x_Q) - A \\
y_R = {\lambda}(x_P-x_Q)-y_P
\end{gathered}
\]
که در آن
\[
\lambda = 
\begin{cases}
\frac{y_Q-y)P}{x_Q-x_P} & P \ne Q,-Q ~~\text{اگر} \\
\frac{3{x_P}^2 + 2Ax_P+1}{2By_P}  & P = Q  ~~\text{اگر}.
\end{cases}
\]
$j$-
پایای این فرم از خم بیضوی برابر با مقدار
$$ j(M_{B,A}) = \frac{256{(A^2-3)}^3}{A^2-4} $$
است که تنها به پارامتر
$A$
بستگی دارد.
\\
همچنین خم تصویری مونت‌گومری در میدان 
$\mathbb{F}_q$
یک خم بیضوی به فرم
$$ M_{B,A} : BY^2Z = X^3+AX^2Z + XZ^2 $$
است که در آن
$A,B \in \mathbb{F}_q$
،
$B \ne 0$
و
$A^2 \ne 4$.
مجموعه نقاطی که روی این خم هستند همراه با نقطه‌ی همانی
$ \infty = (0:1:0) $
گروه نقاط
$M_{B,A}$
را تشکیل می‌دهند.
\\

برای آن‌که بتوانیم خم مونت‌گومری را جایگزین خم وایرشتراس کنیم لازم است تا یک یکیریختی بین آنها پیدا کنیم. چناچه در 
\cite{montgomery_arithmetic}
ذکر شده است اگر در میدان 
$\mathbb{F}_q$
،
$q$
توانی از ۳ نباشد، خم مونت‌گومری 
$M_{B,A}$
با نگاشت گویای
\[
\begin{gathered}
 \phi : M_{B,A} \longrightarrow E \\
(x,y) \mapsto (X,Y) = (B(x+A/3), B^2y)
\end{gathered}
\]
با خم وایرشتراس کوتاه
$$ E : Y^2 = X^3 + (B^2 \frac{1-A^2}{3})X + \frac{B^3A}{3(2A^2/9-1)} $$
یکریخت است.
\\
 همچنین وارون نگاشت 
$\phi$
برابر است با :
\[
\begin{gathered}
{\phi}^{-1} : E \longrightarrow M_{B,A} \\
(X,Y) \mapsto (x,y) = (X/B - A/3, Y/B^2)
\end{gathered}
\]
\\
اگر فرض ‌کنیم
$ E : Y^2 = X^3+aX+b $
یک خم بیضوی باشد در این‌صورت 
$E$
با یک خم مونت‌گومری یکیریخت است اگر و تنها اگر 
$\alpha \in \mathbb{F}_q$
وجود داشته باشد که 
${\alpha}^3+a\alpha+b=0$
و
$\sqrt{3{\alpha}^2 +a} \in \mathbb{F}_q$.
حال اگر 
$\beta = \sqrt{3{\alpha}^2 +a}$
آنگاه نگاشت گویای زیر یک یکریختی بین این دو خم خواهد بود:
\[
\begin{gathered}
{\phi} : E \longrightarrow M_{{3\alpha/\beta},{1/\beta}} \\
(X,Y) \mapsto (x,y) = ((X-\alpha)/\beta, Y/\beta)
\end{gathered}
\]
\\
\\
اعمال جمع و ضرب اسکالر روی نقاط خم بیضوی به فرم مونت‌‌گومری  بااستفاده از یک نگاشت
$x$
انجام می‌شود. این نگاشت روی نقطه‌ی 
$ P = (x:y:z) \in M_{B,A} $
به‌صورت زیر تعریف می‌شود:
\[
\begin{gathered}
x : M_{B,A} \longrightarrow \mathbb{P}^1 \\
P \mapsto 
\begin{cases}
(x:z) & P \ne \infty ~~\text{اگر} \\
(1:0) & P = \infty ~~\text{اگر}
\end{cases}
\end{gathered}
\]
در
\cite{montgomery_speeding}
نشان داده شده است که رابطه‌های
$$ x_{P+Q}(x_P-x_Q)^2x_Px_Q = B(x_Py_Q - x_Qy_Q)^2 $$
$$ 4x_{2P}x_P(x_P^2+Ax_P+1) = (x_P^2 - 1)^2 $$
و
$$ x{P-Q}(x_P-x_Q)^2x_Px_Q = B(x_Py_Q + x_Qy_Q)^2 $$
روی نقاط
$ P,Q \in M_{B,A} $
برقرار است. از این معادلات می‌توان نتیجه گرفت
$$ x_{P+Q}x_{P-Q} = \frac{(x_Px_Q - 1)^2}{(x_P-x_Q)^2} $$
$$ x_{2P} = \frac{(x_P^2-1)^2}{4x_P(x_P^2+Ax_P+1)} $$
و لذا
$$ x_{P+Q}  = \frac{(x_Px_Q-1)^2}{(x_P-x_Q)^2x{P-Q}}$$
$$ x_{2P} = \frac{(x_P^2-1)^2}{4x_P(x_P^2+Ax_P+1)} $$
ار این معادلات می‌توان مولفه‌ی
$x$
نقاط
$P+Q$
و
$2P$
را با مولفه‌های
$x$
نقاط
$P,Q,P-Q \in M_{B,A}$
محاسبه کرد.



































\subsection{ اندازه پارامتر}\LTRfootnote{Parameter Sizes}\label{size_parameter} 
همان‌طور که  قبلا بررسی شد، اعداد اولی که برای ساخت همسانی‌ها از آن استفاده می‌کنیم به فرم
$p = \ell_A^{e_A} \ell_B^{e_B} \cdot f \pm 1$
می‌باشد با این ویژگی که 
$\ell_A^{e_A} \approx \ell_B^{e_B}$.
همچنین یادآوری می‌کنیم که برای داشتن 
$\lambda$
بیت امنیت پساکوانتومی لازم است تا اعداد اول مورداستفاده در طرح امضا به طول
$6 \lambda$
بیت باشند(
\ref{sign_security}
)، درنتیجه اندازه عامل‌های اعداد اول به‌صورت
$\ell_A^{e_A} \approx \ell_B^{e_B} \approx 2^{3 \lambda}$
خواهد بود.
ازآنجا که در طرح امضای خود، خم‌های سوپرسینگولار را در میدان
$\mathbb{F}_{p^2}$
تعریف می‌کنیم درنتیجه اندازه عناصر میدان،
$12\lambda$
بیت طول خواهند داشت.
\\
خم‌هایی که در طرح امضای خود استفاده می‌کنیم به فرم خم‌های مونت‌گومری
\\ 
$By^2 = x^3 + Ax^2 + x$
می‌باشد. از مزیت‌های خم‌ مونت‌گومری می‌توان به محاسبات همسانی‌ها اشاره کرد که فقط به ضریب
$A$
نیاز می‌باشد.
از طرف دیگر، یک نقطه روی خط کامر
\LTRfootnote{Kummer line}
 نیز می‌تواند بوسیله‌ی ضریب
$X$
اش نشان داده شود.
 با این اوصاف،برای نمایش هر عنصر میدان در خم به فرم مونت‌گومری و خط کامر، نیاز به 
$12 \lambda$
بیت می‌باشد.
\\
\\
\textbf{فشرده‌سازی.}
آذردرخش و همکارانش در
\cite{azarderakhsh_key_compress}
نشان داده‌اند که نقاط تابی(که مولد زیرگروه‌های تابی می‌باشند) می‌توانند بوسیله‌ی ضریب‌هایشان فشرده شوند. از آن‌جا که پیاده ‌سازی این روش زیادی کند می‌باشد اخیرا کاستللو و همکارانش در
\cite{costello_efficientkey_compress}
یک الگوریتم جدیدتری ارائه داده‌اند که نسبت به روش آذردرخش  هم سریع‌تر است و هم اندازه کلیدعمومی  آن به نسبت طرح قبلی کوچکتر می‌باشد. در مورد اجرای این الگوریتم می‌توان گفت تقریبا برابر با اجرای یک مرحله از پروتکل اثبات‌دانش‌صفر می‌باشد.
\\
همچنین در بخش قبلی اشاره شد که می‌توانیم  زیرگروه تولید شده توسط یک نقطه تابی را تنها با یک مولد و ضریبش نشان دهیم. از آنجا که نقاط مولد زیرگروه ‌‌ها در طرح ما عمومی می‌باشند درنتیجه می‌توانیم برای نمایش یک زیرگروه تابی تنها از یک  ضریب برای اختصار استفاده کنیم، به عبارت دیگر: 
$$ 
R = mP_A + nQ_A  \xrightarrow[]{m^{-1}}
m^{-1}R=m^{-1}mP_A + m^{-1}nQ_A = P_A + m^{-1}nQ_A = P_A + k Q_A 
$$
با توجه به مطالب بالا برای نمایش
$R$
فقط لازم است که 
$k$
را در اختیار داشته باشیم چون 
$P_A$
و
$Q_A$
عمومی هستند.
\\
در محاسبه ترکیبات  خطی‌، برای فشرده‌‌سازی دو مولد یک گروه تابی، نیاز به سه ضریب  می‌باشد که برای هر ضریب  تقریبا 
$3 \lambda$
بیت نیاز می‌باشد.
\\
\subsubsection{فشرده‌سازی امضا}

به دو روش می‌توانیم، طرح امضای خود را فشرده کنیم:
\begin{itemize}
\item{
فشرده‌سازی کلیدعمومی	
}

\item{
فشرده سازی پاسخ
$\psi(S)$
زمانیکه در مرحله‌ای از الگوریتم امضا،
$ch=1$
انتخاب شده باشد
}
\end{itemize}
لازم به ذکر است، کلیدخصوصی
$S$
 و پاسخ 
$ch = 0$
 یعنی
$(R,\phi(R))$
به دلیل آنکه با ضریب
$3 \lambda$
بیتی قابل نمایش‌اند، لذا نیازی به فشرده‌سازی ندارند.
\\
\begin{itemize}
\item{\textbf{کلیدعمومی. }}
از آنجا که از خم مونتگومری استفاده می‌کنیم بنابراین کلیدعمومی ما به فرم
$pk = (a, x(P_B), x(Q_B), x(P_B-Q_B))$
می‌باشد که 
$a$
بیانگر ضریب
$A$
 در خم عمومی 
 $E / \langle S \rangle$
 می‌باشد. این چهار عنصر میدان به
 $48 \lambda (= 4 \times 12 \lambda)$
 بیت برای ‌نمایش نیاز دارند.	
\\ 
 کلیدعمومی را می‌توانیم با فشرده‌سازی نقاط تابی
 $(\phi({P_B}) , \phi({Q_B}) )$
، که نیاز به سه ضریب
$3 \lambda$
بیتی دارند را فشرده کنیم.
به‌دلیل آنکه مختصات نقاط
$P_B$
و
$Q_B$
از طریق ضرایب فشرده‌شان قابل تولید می‌باشد بنابراین
نیازی به ضریب
$X$
نقطه‌ی
$\phi(P_B-Q_B)$
نمی‌باشد. 
بنابراین به‌طورکلی در کلیدعمومی برای نمایش خم،
$12\lambda$
بیت و برای مولدها نیز
$9 \lambda$
بیت نیاز داریم که جمعا 
$21 \lambda$
بیت می‌شود.
% ----------------------------------------------
\item{\textbf{کلیدخصوصی.}}
کلیدخصوصی
$S$
می‌تواند تنها با یک ضریب
$n$
که نیاز به
$3 \lambda$
بیت می‌باشد ذخیره شود. دلیل این امر هم این است که کلیدخصوصی
$S$
از مرتبه‌ی
$\ell_A^{e_A}$
می‌باشد و 
$ S = P_A + [n] Q_A $.
% ----------------------------------------------
\item{\textbf{امضا.}} 
برای هر مرحله‌ی
$i$
ام از پروتکل اثبات‌دانش‌صفر، امضا شامل چندتایی

$(com_i, ch_{i,j}, h_{i,j}, resp_{i,J_i})$
می‌باشد. بنابراین:
\begin{itemize}
\item{
	هر تعهد شامل دو خم
	$(E_1,E_2)$
	می‌باشد که هر کدام از این خم‌ها به یک عنصر میدان که همان ضریب 
	$A$
	می‌باشد، نیاز دارند.

}
\item{
یک بیت برای نمایش بیت چالشی
$ch_{i,0}$
نیازاست. البته قابل ذکر است که اگر مقدار
$ch_{i,0}$
را داشته باشیم نیازی به ارسال
$ch_{i,1}$
نمی‌باشد، دلیل این امر هم تساوی 
$ch_{i,1} = 1 - ch_{i,0}$
می‌باشد.
}
\item{
چنانچه در
\ref{sign_security}
 توضیح داده شده است، برای هش
$h_{i,j} = G(resp_{i,J_I})$
نیز به
$3 \lambda$
بیت فضا نیاز می‌باشد.
\\
 ذکر این نکته نیز لازم است که با 
$resp_{i,J_i}$
، می‌توان
$h_{i,j}$
را محاسبه کرد و بنابراین نیازی به ارسال این هش وجود ندارد.
}
\item{
براساس بیت چالشی
$J_i$
جواب‌های متفاوتی خواهیم داشت و ازاین‌رو طول بیت متفاوتی نیز برای ذخیره‌سازی لازم خواهد بود. اگر
$J_i = 0$
آنگاه پاسخ موردنظر
$(R,\phi(R))$
خواهد بود که دراین صورت با توجه به وجود مولدهای عمومی، بدون هیچ هزینه محاسباتی نیاز به 
$3 \lambda$
بیت برای ذخیره‌سازی لازم خواهد بود.
اگر
$J_i = 1$
آنگاه پاسخ،
$\psi(S)$
است که به
$12 \lambda$
بیت به‌عنوان یک عنصر میدان لازم خواهد بود که با فشرده سازی به
$3 \lambda$
بیت تقلیل می‌یابد.
}
\end{itemize}
   
\end{itemize}	
در مجموع، برای هر مرحله از اثبات دانش صفر تقریبا به طورمتوسط به 
$$24 \lambda + 1  + 3 \lambda + \frac{3\lambda + 12\lambda}{2} \approx 34.5\lambda$$
بیت فضا بدون فشرده سازی نیازاست که با فشرده‌سازی تقریبا به‌طور متوسط به 
$$24 \lambda + 1 + 3 \lambda + 3 \lambda \approx 30\lambda$$
بیت نیاز خواهد بود.
\\

اگرچه برای تامین
$\lambda$
بیت امنیت پساکوانتومی کفایت می‌کند تا پروتکل اثبات دانش صفر،
$\lambda$
بار تکرار شود اما به‌دلیل آنکه هش چالش‌ها دربرابر الگوریتم گراور
\cite{grover}
آسیب‌پذیر نباشد(بخش ۵٫۳)، لازم است که پروتکل امضا،
$2\lambda$
بار پروتکل اثبات دانش صفر را تکرار کند. با این اوصاف در کل، امضا تقریبا به‌طورمتوسط
$69{\lambda}^2(=2\lambda \times 34.5\lambda)$
بیت در حالت عادی و 
$60{\lambda}^2$
بیت درحالت فشرده‌سازی لازم دارد.
\\
به‌عنوان مثال برای دست‌یابی به 
$128$
بیت امنیت پساکوانتومی(تعداد بیتی که در حالت پساکوانتومی ایمن باشد) برای طرح امضای ارائه شده، به‌طور متوسط به
$48\lambda = 6144$
(
$2688$
در حالت فشرده) بیت برای کلیدعمومی، 
$3\lambda = 384$
بیت برای کلیدخصوصی و
\\
$69{\lambda}^2 = 1,130,496$
($122,880$
برای حالت فشرده) بیت برای امضا لازم است.
\subsubsection{سنجش}
دراین قسمت می‌خواهیم سایز پارامترهای لازم در طرح خود را با سایر طرح‌های امضای پساکوانتومی مقایسه می‌کنیم.
\\
همان‌طور که از جدول زیر قابل مشاهده‌ است، طرح امضای معرفی شده در این پایان‌نامه دربرابر سایر طرح‌های امضای پساکوانتومی موجود دارای کلید با طول سایز کوچکتر می‌باشد. البته قابل ذکر است که گونه‌هایی از طرح امضای مرکل وجود دارد که دارای طول کلید کوچکتری(۳۲ بیت) با همان درجه امنیت می‌باشد اما؟؟.

\begin{center}
	\begin{table}[h]\label{tbl:comparison_signature_scheme}
	\caption{
	سنجش سایز پارامترها(به بایت) در طرح‌های امضاهای پساکوانتومی 
	\\
	متفاوت در سطح امنیتی ۱۲۸ بیت کوانتومی
	}	
	\begin{tabular}{ r | c | c | c }
		طرح امضا & سایز کلیدعمومی & سایز کلیدخصوصی & سایز امضا  \\ 
		\hline
		هش مبنا & 1,056 & 1,088 & 41,000 \\ 
		کد مبنا & 192,192 & 1,400,288 & 370 \\ 
		مشبکه مبنا & 7,168 & 2,048 & 5,120 \\ 
		حلقه مبنا & 7,168 & 4,608 & 3,488 \\ 
		چندمتغیره مبنا & 99,100 & 74,000 & 424 \\ 
		\hline
		همسانی مبنا & 768 & 48 & 141,312 \\ 
		همسانی مبنای فشرده & 336 & 48 & 122,880 \\   
	\end{tabular}
\end{table}
\end{center}




