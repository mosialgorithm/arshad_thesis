\section{اثبات دانش صفر هویت}\label{ZKPOI}

به‌طور رسمی، یک سیستم اثبات دانش صفر یک رویه است که طی آن پگی(به‌عنوان شخص اثبات‌کننده)، ویکتور(به‌عنوان شخص تاییدکننده) را متقاعد می‌کند که به یک حقیقت معین  
\RTLfootnote{
حقیقت می‌تواند هویت اثبات‌کننده (پگی) ‌باشد.
}
اشراف دارد بطوریکه هیچ اطلاعات اضافی نسبت به دانش خود در اختیار ویکتور قرار نمی‌دهد تا بدین‌منظور خود ویکتور نتواند به عنوان یک مدعی، دیگران را متقاعد کند که به حقیقت مورد بحث اشراف دارد. برای توضیح بیشتر این پروتکل مثالی ارائه می‌کنیم.
\begin{example}\label{ex_zkp}\textbf{اثبات دانش صفر}
	
فرض کنید دو لیوان شفاف در اختیار داریم که یکی حاوی آبِ خالص و دیگری حاوی آب با مخلوطی شفاف می‌باشد که فقط پگی فرق این دو لیوان را می‌داند. حال برای آن‌که پگی به ویکتور ثابت کند که دانش لازم را برای تشخیص لیوان حاوی آب خالص و لیوان آب ناخالص را در اختیار دارد می‌بایست به چالش‌هایی که از طرف ویکتور مورد سوال قرار می‌گیرد به درستی جواب بدهد. ویکتور برای اطمینان از اینکه پگی واقعا دانش لازم این اثبات را می‌داند می‌تواند چالش‌های خود را چندین بار تکرار کند و اگر پگی در تمامی چالش‌ها به درستی جواب بدهد آنگاه مطمئن می‌شود که پگی دانش لازم را دراختیار دارد. پگی برای آن‌که مستقیما دانش خود را افشا نکند لیوان حاوی آب خالص را به ویکتور نشان نمی‌دهد و در عوض مراحل زیر به تعداد مشخصی تکرار می‌شود
\begin{enumerate}
	\item 
	ابتدا پگی به عنوان اثبات‌کننده ادعا می‌کند که مکان لیوان‌ حاوی آب ناخالص  را می‌داند.
	\item 
 پگی چشمان خود را با چشم‌بند می‌بندد و سپس	ویکتور به‌عنوان یک  چالش ، یا جای دو ظرف آب را باهم جا‌به‌جا می‌کند یا بدون تغییر آن‌ها آماده پاسخ چالش خود می‌شود
	\item 
	پگی چشم‌بند را از چشمان خود برمی‌دارد و با اتکا به دانشی که در اختیار دارد مشخص می‌کند که آیا جای این دو لیوان عوض شده است یا خیر
	\item 
	اگر پگی به درستی تشخیص دهد که جای لیوان‌ها عوض شده است یا خیر  آنگاه ویکتور برای اطمینان از شانسی نبودن جواب پگی می‌تواند بار دیگر مراحل را با همکاری پگی تکرارکند اما اگر حتی یک بار پگی به اشتباه جواب چالش پگی را بدهد آنگاه ویکتور با اطمینان ادعای پگی را نمی‌پذیرد.
\end{enumerate}~
\\
ذکر این نکته لازم است که اگر پگی به صورت شانسی به چالش جواب بدهد به احتمال
$1/2$
به طور صحیح  جواب داده است ، حال اگر  که رویه اثبات به تعداد 
$n$
بار تکرار شود آنگاه به احتمال 
$1 / 2^n$
 پگی به صورت شانسی جواب چالش‌ها را  به درستی داده است ، همان‌گونه که مشخص است تقریبا محال است که همچین اتفاقی رخ دهد و پگی قادر باشد که تمام جواب‌ها را به صورت شانسی جواب داده باشد. بنابریان ویکتور با تکرار رویه اثبات و جواب صحیح پگی در هر مرحله، کاملا قانع می‌شود که پگی به دانش ادعا شده اشراف دارد.

\end{example}~
\\
برای پیاده‌سازی پروتکل‌اثبات‌دانش‌صفر به صورت ریاضی، از همسانی بین  خم‌های سوپرسینگولار استفاده می‌کنیم.  
\\
در طرحی که خواهان ارائه آن هستیم به خم‌های سوپرسینگولار با درجه‌ای هموار
\RTLfootnote{
در مبحث خم‌های سوپرسینگولار، ساخت خم‌های با درجه هموار آسان می‌باشد و با استفاده از این خم‌ها می‌توان تعداد زیادی همسانی بین آنها ساخت که خیلی سریع قابل محاسبه هستند.
}
 نیاز می‌باشد بنابراین ابتدا  عدد اولی به فرم 
$p = \ell_A^{e_A} \ell_B^{e_B} \cdot f \pm 1$
را انتخاب می‌کنیم که 
$\ell_A$
و
$\ell_B$
اعداد اول کوچک (معمولا ۲ و ۳) می‌باشند با این خاصیت که طول ارقام
$\ell_A^{e_A}$
و
$\ell_B^{e_B} $
برابر باشد (در بخش ۶ به طور مفصل آن را بررسی خواهیم کرد) و همچنین  
$f$ % section 6 dont defind until now !!!!!!!!!!!!!!!!!!!!!!!!
یک عامل کوچک است که باعث می‌شود 
$p$
یک عدد اول شود(از آنجا که 
$\ell_A^{e_A}$
و
$\ell_B^{e_B} $
دیگر اول نیستند). در ادامه ‌با روش بروکر 
\cite{broker}
، یک خم سوپرسینگولار 
$E$
را روی میدان 
$F_{p^2}$
با مرتبه‌ی 
$(\ell_A^{e_A} \ell_B^{e_B})^2$
 به دست می‌آوریم. سپس دو زیرگروه 
$E[\ell_A^{e_A}]$
و
$E[\ell_B^{e_B}]$ 
که مولدهای آن به ترتیب به صورت زوج نقاط
$ \langle P_A,Q_A \rangle $
و
$ \langle P_B,Q_B \rangle  $
می‌باشند را روی خم 
$E$
محاسبه می‌کنیم.
\\
طرح اثبات دانش صفر براساس همسانی‌ها مطابق شکل 
\ref{fig:zkp}
 صورت می‌گیرد. پگی به عنوان اثبات کننده، نقطه 
$S$
که تولیدکننده هسته همسانی 
$\phi : E \longrightarrow E/ \langle S \rangle $
می‌باشد را به عنوان دانش  خود، به صورت مخفی نزد خود نگه  می‌دارد. 
و بر این اساس، کلید خصوصی و کلیدعمومی پگی به‌صورت زیر معرفی می‌شود :

\begin{itemize}
\item {\textbf{کلید خصوصی :} }
هسته همسانی
$\phi$
یعنی
$S$

\item {\textbf{کلید عمومی :} }
خم بیضوی
$E/ \langle S \rangle$
و تصویر نقاط  
$P_B$ 
و
$Q_B$
یعنی
$\phi(P_B)$
و	
$\phi(Q_B)$	
\end{itemize}~ 
% ==============================================================================================
% figure 1 for demonstrate zer-konwledge proof of identity
\begin{figure}[H] 
	\begin{center}
		
		\begin{tikzcd}
			E \arrow[r, "\phi"] \arrow[d, "\psi"] & E/ \langle S \rangle \arrow[d, "{\psi}' "] \\
			E/ \langle R \rangle \arrow[r, "{\phi}' "] & E/\langle R,S \rangle
		\end{tikzcd}
		
		\caption{
			هر فلش با همسانی و و هسته‌اش نشانه گذاری شده است    
		}
		\label{fig:zkp}
		
	\end{center}
\end{figure}
% ==============================================================================================
\vskip 0.5in
حال پگی برای آن که به ویکتور (تاییدکننده) ثابت کند که دانش 
$\langle S \rangle $
را می‌داند ، مراحل زیر به ترتیب انجام می‌شود:
\begin{enumerate}
	
\item {
\begin{itemize}
\item 
 نقطه تصادفی 
$R$
را از مرتبه‌ی
$\ell_B^{e_B}$
انتخاب می‌کند.
	
\item 
 همسانی
$\psi : E \rightarrow E / \langle R \rangle$
را محاسبه می‌کند.
\item 
در ادامه همسانی‌های
${\phi}' : E / \langle R \rangle \rightarrow  E / \langle R,S \rangle $
را باهسته
$\langle \psi(S) \rangle$
و همچنین همسانی
${\psi}' : E / \langle S \rangle \rightarrow  E / \langle R,S \rangle $
را با هسته‌ی
$\langle \phi(R) \rangle$
 از طریق فرمول ولو محاسبه می‌کند.
 \item 
 پس از محاسبات بالا ، پگی تعهد 
 \LTRfootnote{commitment}
 $com = (E_1 , E_2)$
را برای ویکتور ارسال می‌کند که
 $E_1 = E / \langle R \rangle$
 و
 $E_2 = E / \langle R,S \rangle$
‌ می‌باشند.
 
\end{itemize}	
} % end item

\item 
ویکتور به طور تصادفی  بیت چالشی 
$ch \in \{0,1\}$
را انتحاب و برای پگی ارسال می‌کند.
\item 
پگی پاسخ 
$resp$
را برای ویکتور ارسال می‌کند :
\begin{itemize}
	\item
	اگر 
	$ch = 0$
	آنگاه
	$resp = (R,\phi(R))$
	
	\item
	اگر 
	$ch = 1$
	آنگاه
	$resp = \psi(S)$
	
\end{itemize}
\item {
\begin{itemize}
	\item 
	اگر
	$ch = 0 $
	، ویکتور ابتدا بررسی می‌کند که آیا
	$R$
	و
	$\phi(R)$
	هردو از مرتبه‌ی 
	$\ell_B^{e_B}$
	هستند یا خیر . در ادامه بررسی می‌کند که آیا این دو، هسته‌ی همسانی‌های
	$E \rightarrow E_1$
	و 
	$E/ \langle S \rangle \rightarrow E_2$
	را تولید می‌کنند یا خیر.
	\item 
	اگر 
	$ch = 1 $
	، ویکتور ابتدا بررسی می‌کند که آیا  
	$\psi(S)$
	از مرتبه‌ی
	$\ell_A^{e_A}$
	می‌باشد یا خیر و همچنین درادامه بررسی می‌کند که آیا هسته‌ی همسانی 
	$E_1 \rightarrow E_2$
	را تولید می‌کند یا خیر.
\end{itemize}
} % end item
\end{enumerate}~
\remark
قابل ذکر است که در مرحله‌ی ۴، اگر نتیجه‌ی بررسی تاییدکننده مثبت باشد، ادعای اثبات‌کننده تایید و در غیر اینصورت ادعای وی رد می‌شود.	
\\
\remark
همچنین قابل ذکر است که رابطه‌ی زیر همواره برقرار است:
$$
\frac{(E/\langle S \rangle)}{\langle \phi(R) \rangle} =
\frac{E}{\langle R,S \rangle}  = 
\frac{( E/ \langle R \rangle)}{\langle \psi(S) \rangle} 
$$


% ==============================================================================================
% fugure 2 for demostrate DSSP assumption for chaaleng ????
\begin{figure}[H]
\centering
\begin{subfigure}[b]{0.4\linewidth}
\caption{b = 0}
\label{fig:sub1}
\begin{tikzcd}
	E \arrow[r, dashrightarrow , "\phi"] \arrow[d, "\psi"] & E/ \langle S \rangle \arrow[d, "{\psi}' "] \\
	E/ \langle R \rangle \arrow[r, dashrightarrow ,"{\phi}' "] & E/\langle R,S \rangle
\end{tikzcd}

\end{subfigure}


\begin{subfigure}[b]{0.4\linewidth}
\caption{b = 1}
\label{fig:sub2}
\begin{tikzcd}
	E \arrow[r, dashrightarrow , "\phi"] \arrow[d, dashrightarrow , "\psi"] & E/ \langle S \rangle \arrow[d, dashrightarrow , "{\psi}' "] \\
	E/ \langle R \rangle \arrow[r, "{\phi}' "] & E/\langle R,S \rangle
\end{tikzcd}
	
\end{subfigure}%

	% -------------------------------------------------------------------
	\caption{
			همسانی‌های مخفی با خط‌های مقطع نمایش داده شده است. خط های توپر نمایش دهنده همسانی‌هایی می‌باشد که پگی نسبت به چالش انجام شده ظاهر می‌کند. با این حال همسانی‌های ظاهرشده هیچ اطلاعاتی درباره همسانی مخفی
			$\phi$
			افشا نمی‌کند.	
	}
	\label{fig:challenge}
\end{figure}~
% ==============================================================================================

 برای دست‌یابی به 
$\lambda$
بیت امنیت، لازم است که عدد اول
$p$
دقیقا 
$6\lambda$
بیت باشد(دلیل این امر در بخش ۶ ذکر شده است) و پروتکل 
$\lambda$
بار تکرار شود. اگر ویکتور(تاییدکننده) تمام
$\lambda$
بار تکرار پروتکل را با موفقیت تایید کند آنگاه پگی(اثبات‌کننده) توانسته است ادعای خود مبنی بر دانش کلیدخصوصی
$S$
را برای ویکتور اثبات کند، در غیراینصورت ادعا توسط ویکتور موردقبول قرار نمی‌گیرد.


لازم به ذکر است که در هربار اجرای این طرح، نقطه
$R$
کاملا به صورت تصادفی انتخاب می‌شود و براساس این نقطه، شکل
\ref{fig:zkp}
حاصل می‌شود و بنابراین همسانی‌ها و خم‌های هر بار اجرای این پروتکل با دفعه قبل کاملا فرق دارد و فقط خم 
$E / \langle S \rangle $
ثابت باقی می‌ماند.
\\
 همچنین چنان‌‌که در بخش 
6
خواهیم دید اگر همسانی‌های 
$\psi$
و
${\psi}'$
و
${\phi}'$
همزمان معلوم شوند، دانش  
$S$
 قابل کشف است بنابراین برای جلوگیری از افشای 
$S$
، در هر بار تکرار رویه اثبات، درمقابل چالش 
 $ch = 0 $
 همسانی‌های 
 $\psi$
 و
 ${\phi}'$ 
 عمومی می‌شوند و درمقابل چالش
 $ch = 1$
 تنها همسانی 
 ${\phi}'$ 
 عمومی می‌شود تا طرح ما ایمن باقی بماند.
 
