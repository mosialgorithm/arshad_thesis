\documentclass[12pt,a4paper]{article}
%\documentclass[12pt]{book}
\let\latinrm\mathrm


\usepackage{amsthm,amsmath,amssymb,mathtools}
\usepackage{lipsum}
\usepackage{algorithm}
%\usepackage{algorithmic}
\usepackage[noend]{algpseudocode}
\usepackage{tikz-cd}
\usetikzlibrary{decorations.pathmorphing}
\usepackage{subcaption}
\usepackage{hyperref}
\usepackage{cite}
\renewcommand{\algorithmiccomment}[1]{$\triangleright$ #1}
\usetikzlibrary{matrix}
%\usepackage[demo]{graphicx}
% \usepackage{caption}


\linespread{1.5} 

\usepackage{xepersian}
\settextfont{XBZar}
\setdigitfont{XBZar}

%\title{امضای دیجیتال مقاوم کوانتومی بر اساس همسانی های بین خم های سوپرسینگولار}
%\author{مصطفی قربانی
%	\\[1cm]{ استاد راهنما: دکتر حسن دقیق}}
%\author{مصطفی قربانی}
\date{}


\theoremstyle{plain}
\newtheorem{theorem}{قضیه}
\newtheorem{lemma}{لم}
\newtheorem{proposition}{گزاره}
\theoremstyle{definition}
\newtheorem{definition}{تعریف}[section]
\newtheorem{example}{مثال}
\newtheorem{prob}{سوال}
\theoremstyle{remark}
\newtheorem{corollary}{نتیجه}
\newtheorem{remark}{ملاحظه}


\begin{document}
	%\maketitle
	%\tableofcontents
	%\listoffigures
	%\listoftables	
	%\chapter{پیش نیازها}\label{prerequesties}	
	%\begin{theorem}\end{theorem}


\section{امضای کور غیرقابل انکار}\label{undeniable_blind_sig}
\LTRfootnote{Undeniable Blind Signature}

طرح امضای کور پروتکلی است که طی آن درخواست‌کننده بدون افشای محتوای سند از امضاکننده در خواست می‌کند تا سند را امضا کند.در سال ۱۹۸۲ اولین بار چام طرح امضای کور رامعرفی کرد.
\cite{chaum@blind}
این طرح براساس مسئله 
$RSA$
بنا شده است.
\cite{rivest@rsa}
از آنجا که اکثر طرح‌های امضای کور و تغییرات آن براساس سختی مسائل متفاوتی از جمله مسئله لگاریتم گسسته
\LTRfootnote{Discrete Logarithm Problem (DLP)}
 ، مسائل زوجیت‌مبنا 
\LTRfootnote{pairing-based problems}
  و مسائل مشکبه‌مبنا
\LTRfootnote{lattice-based problems}
   ارائه شده است 
\cite{discrete@blind, lattice@blind, pairing@blind}
، ولی تمام این طرح‌ها یک مشکل اساسی دارند و مشکل این است که در برابر متخاصم کوانتومی ایمن نمی‌باشند. امضاهای کور معرفی شده توسط چام 
\cite{chaum@blind}
، کامنیش  
\cite{discrete@blind}
و ژانگ‌و‌کیم 
\cite{pairing@blind}
به دلیل الگوریتم شور 
\RTLfootnote{در زمان چندجمله‌ای مسائل لگاریتم گسسته و تجزیه‌اعداد را در کامپیوترهای کوانتومی حل می‌کند}
 در برابر حملات کوانتومی ایمن نیستند.
چنان‌که در  
\cite{shamir@quantum}
نشان داده شده است ، امضای کور مشبکه‌مبنای معرفی شده توسط روکرت  
\cite{lattice@blind}
که از مدل فیات‌شمیر 
\cite{fiat@prove}
استفاده می‌کند در برابر مدل اوراکل تصادفی کوانتومی ایمن 
\LTRfootnote{quantum random oracle model}
نمی‌باشد.

امضای کور هر دو ویژگی ناشناس‌بودن
\LTRfootnote{anonymity}
 و احرازهویت 
\LTRfootnote{authentication}
 را در خود دارد.
\cite{untraceable, untraceability}
در‌نتیجه این طرح در بسیاری از پروتکل‌های حفظ حریم‌خصوصی 
\LTRfootnote{privacy-preserving}
ازجمله پول‌الکترونکی 
\LTRfootnote{e-cash}
و رای‌گیری‌الکترونیکی
\LTRfootnote{e-voting}
استفاده می‌شود.
\cite{e_vote@mobile, e_vot@net }
چنان‌چه در ابتدا گفته‌شد امضاکننده هیچ کنترلی بر محتوای سندی که قرار است امضا شود را ندارد ، علاوه‌بر‌این امضاکننده هیچ کنترلی در نحوه استفاده از امضا را هم ندارد. با این اوصاف احساس می‌شود اعطای درجه‌ای از کنترل به امضاکننده نیاز است. یک از راه‌های ممکن آن است که امضاکننده و درخواست کننده(امضا) روی بخشی از محتوای سند توافق کنند. این راه توسط تکنیکی که آبه و فوجیساکی در 
\cite{date@blind}
ارائه کرده‌اند قابل دستیابی می‌باشد.

راه دیگر آن است که این اختیار به امضاکننده داده شود تا تصمیم بگیرد چه‌کسی مجاز به تایید امضا می‌باشد.این روش ؟؟؟؟.
طرح امضای غیرقابل‌انکار معرفی‌شده توسط چام و ون‌آنترپن 
\cite{undeniable_chaum}
دقیقا مطالب بالا
\RTLfootnote{در یک طرح امضای غیرقابل‌انکار، امضاکننده تصمیم می‌گیرد تا چه کسی امضا را تایید کند}
 را دربرمی‌گیرد. 
 
بنابراین مطلوب است طرحی داشته باشیم که ناشناس‌بودن و تاییدسازی‌کنترل‌شده را درخود داشته باشد که ویژگی‌های هر دو طرح امضای کور و امضای غیرقابل‌انکار را برآورده کند.
% چنین طرحی می تواند طراحی شود اما مشخص نیست.
در سال ۱۹۹۶ ، ساکوری و یامانه 
\cite{1st_blind_undeniable_sig}
یک طرح امضای کور غیرقابل‌انکار را براساس مساله لگاریتم گسسته ارائه دادند. چنان‌که در 
\cite{undeniable_chaum}
گفته‌شده‌است با این تکنیک می‌توان یک طرح امضای کور غیرقابل‌انکار بر اساس مسئله آراس‌آ 
\LTRfootnote{RSA}
طراحی کرد. ذکر این نکته لازم است که تمام این طرح‌ها در برابر حملات کوانتموی ایمن نیستند.

در این پایان‌نامه در نظر داریم یک طرح امضای کور غیرقابل‌انکار مقاوم کوانتومی بر اساس سختی مسائل همسانی روی خم‌های بیضوی سوپرسینگولار ارائه کنیم.


سوخارو و همکارانش در 
\cite{soukharev}
پیشنهادی درباره‌ی ساخت یک طرح امضا با تاییدکننده معین‌شده براساس سختی مسائل همسانی که مقاوم کوانتومی نیز می‌باشد ارائه کرده است.آنها همچنین یک ساخت عمومی از طرح رمزگذاری تایید‌اعتبار کلید نامتقارن ؟؟؟ را نشان داده‌اند.جائو و سوخارو در
\cite{undeniable}
یک طرح امضای غیرقابل‌انکار همسانی‌مبنا ارائه کرده‌اند.در این پایان‌نامه قصد داریم طرح جائو‌ و سوخارو را به یک طرح امضای کور غیرقابل‌انکار توسعه دهیم.

\subsection{تعریف استاندارد}\label{blind_def}\LTRfootnote{Formal Definition}

انتظار می‌رود طرح امضای کور غیرقابل‌انکار
$(UBSS)$\LTRfootnote{Undeniable Blind Signature Scheme }
، ویژگی‌های طرح امضای غیرقابل‌انکار و طرح امضای کور را همزمان داشته باشد.در نتیجه این طرح باید ویژگی‌های ناخوانابودن محتوای پیام‌اولیه(قبل از امضا)
\LTRfootnote{anonoymity of the message origination}
 و تاییدسازی کنترل شده 
\LTRfootnote{controlled verification}
 را دارا باشد.

\textbf{تعریف۱.}\label{definition_ubss}
طرح امضای کور غیرقابل‌انکار ، یک طرح امضای تعاملاتی است که‌بوسیله چندتایی زیر معرفی می‌شود:

$$ \mathcal{UBSS} = \big( KeyGen , Blind , Sign , Unblind , Check , \mathcal{CON} , \mathcal{DIS} \big) $$

\begin{enumerate}
	\item 
الگوریتم تولید کلید تصادفی 
$KeyGen$
، پارامتر امنیتی
$1^\lambda$
را به عنوان ورودی گرفته و زوج کلیدهای
$(vk,sk)$
را که به عنوان کلیدتاییدساز و کلید‌مخفی نامیده می‌شوند، به عنوان خروجی تولید می‌کند. شکل شماتیک این الگوریتم به‌صورت زیر می‌باشد:
$$ (vk,sk) \longleftarrow KeyGen(1^{\lambda}) $$

\item 
الگوریتم کورسازی تصادفی
$Blind$
، پیام 
$m$
را به عنوان ورودی گرفته و خروجی آن کورشده‌ی پیام، یعنی
$m'$
می‌باشد. شکل شماتیک این الگوریتم به شکل زیر می‌باشد که 
$r$
کاملا به صورت تصادفی توسط الگوریتم ساخته
می‌شود:
$$ m' \longleftarrow {_{r}Blind(m)} $$

\item
الگوریتم امضای قطعی یا تصادفی 
$Sign$
، کلید مخفی
$sk$
و پیام
$m$
را به عنوان ورودی گرفته و امضای
$\sigma$
را به عنوان خروجی تولید می‌کند. این الگوریتم را می‌توان به صورت زیر نشان داد:
$$ \sigma \longleftarrow Sign_{sk}(m) $$

\item 
الگوریتم شفاف‌ساز قطعی 
$Unblind$
،امضای کور
$\sigma'$
و عددتصادفی 
$r$
(انتخاب شده توسط الگوریتم کورسازی) را به عنوان ورودی گرفته و امضای شفاف
$\sigma$
را به عنوان خروجی تولید می‌کند. این الگوریتم را می‌توان به شکل زیر نمایش داد:
$$ \sigma := Unblind_r(\sigma') $$

\item 
الگوریتم قطعی بررسی 
$Check$
، پیام 
$m$
، امضای شفاف
$\sigma$
و زوج کلیدهای
$(vk,sk)$
را به عنوان ورودی گرفته و بیت 
$b$
را به عنوان خروجی تولید می‌کند.
$b=1$
به معنای آن است که امضا متعلق به پیام می‌باشد و 
$b=0$
نیز به این معناست که امضا غیرمعتبر می‌باشد. این الگوریتم به‌صورت زیر قابل نمایش است:
$$ b := Check_{(vk,sk)}(m,\sigma) $$

\item
پروتکل تایید
$\pi_{con}$
توسط امضاکننده اجرا می‌شود تا تاییدکننده اطمینان یابد که امضا معتبر است.

\item 
پروتکل انکار
$\pi_{dis}$
نیز توسط امضاکننده اجرا می‌شود و تاییدکننده متقاعد می‌شود که امضا نامعتبر است.
\end{enumerate}

برای هر زوج کلید 
$(vk,sk)$
که توسط الگوریتم 
$KeyGen(1^{\lambda})$
تولید می‌شود و همچنین هر
$m$
از میان فضای پیام و هر عددتصادفی
$r$
که توسط الگوریتم 
$Blind$
تولید شده است، باید تساوی زیر برقرار باشد:
$$ Check_{(vk,sk)}(m,Unblind_r(Sign_{sk}(_{r}Blind(m)))) = 1 $$


علاوه‌براین، اگر الگوریتم امضا قطعی باشد آنگاه می‌توان فرض کرد اثر مراحل الگوریتم‌های کورسازی-امضا-شفافیت روی پیام دقیقا مشابه اجرای مستقیم الگوریتم امضا روی پیام می‌باشد. برای درک این مطلب آن را به صورت زیر نمایش می‌دهیم:
$$ Unblind_r(Sign_{sk}(_{r}Blind(m))) = Sign_{sk}(m) $$

\newpage
\subsection{کارکرد UBSS}\label{working_ubss}
\LTRfootnote{Workinf of UBSS}

برای درک بهتر نقش الگوریتم‌های گفته شده در بخش قبلی، پروتکل را به صورت کامل اجرا می‌کنیم.

در ابتدا امضاکننده یک پارامتر امنیتی 
$\lambda$
را انتخاب و الگوریتم 
$KeyGen(1^\lambda)$
را برای به دست آوردن زوج کلید
$(vk,sk)$
اجرا می‌کند.کلید امضای
$sk$
به صورت مخفی پیش امضاکننده حفظ می‌شود و کلید تاییدساز
$vk$
توسط امضاکننده منتشر می‌شود.
$m$
پیامی است که درخواست‌کننده خواهان امضای آن به صورت ناخوانا است؟؟. به این منظور ، درخواست‌کننده ابتدا 
$m$\RTLfootnote{پیام خوانا}
 را با اجرای الگوریتم 
$Blind(m)$
به 
$m'$\RTLfootnote{پیام ناخوانا}
تبدیل می‌کند.
\RTLfootnote{
در زمان اجرای الگوریتم ، یک انتخاب تصادفی
$r$
توسط خود الگوریتم تولید می‌شود.
}
درادامه درخواست‌کننده 
$m'$
را به همراه شناسه هویتی خود
$Id_R$
، ارسال می‌کند. امضاکننده ابتدا شناسه درخواست‌کننده را تایید (
\ref{remark_1}
) و سپس الگوریتم 
$Sign_{sk}$
را روی
$m'$
اجرا می‌کند تا امضای کور
$\sigma'$
به دست آید. دریافت‌کننده پس از دریافت امضای کور از امضاکننده ، توسط الگوریتم
$Unblind$
 و مقدارتصادفی 
$r$
انتخاب شده در مرحله کورسازی، امضا را از حالت کور خارج کرده وسپس زوج پیام اصلی و امضای شفاف
$(m,\sigma)$
پیام رابرای بخش تایید؟؟ ارسال می‌کند.

هربخشی که خواهان تایید امضا باشد، شناسه خود
$Id_V$
را به همراه زوج پیام و امضا 
$(m,\sigma)$
برای امضاکننده ارسال می‌کند. امضاکننده در ابتدا شناسه تاییدکننده را بررسی می‌کند(
\ref{remark_1}
) آنگاه اگر
$Id_V$
یک شناسه معتبر در میان تاییدکنندگان احراز شده(مجاز) نباشد، امضاکننده از ادامه ارتباط خودداری می‌کند. در غیراینصورت الگوریتم بررسی 
$Check$
را اجرا می‌کند. اگر خروجی این الگوریتم معتبر باشد آنگاه پروتکل تایید
$\mathcal{CON}$
توسط امضاکننده آغاز می‌شود؛ درغیراینصورت پروتکل انکار
$\mathcal{DIS}$
اجرا می‌شود(شکل 
\ref{fig:ubss}
 تمام مفاهیم طرح 
$UBSS$
را نشان می‌دهد).

% ==============================================================================================
% figure for demonstrate full of information in UBSS
\begin{figure}[H] 
	\begin{center}
		
		\begin{tikzcd}[row sep=5em,column sep=5em]
			& \color{red} \fbox{امضاکننده} \arrow[dl,red,"\sigma'" ,red] 
			         \arrow[dr,red,leftrightsquigarrow , "\mathcal{CON} / \mathcal{DIS}" ,red] & \\
			\color{blue} \fbox{درخواست‌کننده} \arrow[rr,blue, "{(m,\sigma)}",blue ] 
						\arrow[ur ,blue, shift left , "Id_R ||m' ",blue] &
			& \color{green} \fbox{تاییدکننده} \arrow[ul ,green , shift left , "Id_V|| {(m,\sigma)}" ,green]
		\end{tikzcd}
		
		\caption{ اطلاعات کامل طرح امضای کور غیرقابل‌انکار}
		\label{fig:ubss}
		
	\end{center}
\end{figure}
% ==============================================================================================

\textbf{توجه۱.}\label{remark_1}
در این پایان‌نامه عمدا چگونگی احراز هویت  بین درخواست‌کننده و تاییدکننده با امضاکننده را مشخص نمی‌کنیم. این امر مستلزم آشنایی با احراهویت‌ متقابل می‌باشد. این طرح در
\cite{boneh@secure , goorden@secure}
به‌صورت کامل آورده شده است که در مقابل حملات کوانتومی نیز ایمن می‌باشند.

\subsection{ویژگی‌ها}\label{ubss_properties}\LTRfootnote{Properties}

% ======================================================================
% Refrences
% ======================================================================
\newpage
\setLTRbibitems
% \resetlatinfont
\bibliographystyle{plain}
\bibliography{ref.bib}

\end{document}