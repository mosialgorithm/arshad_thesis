\documentclass[12pt,a4paper]{article}
%\documentclass[12pt]{book}
\let\latinrm\mathrm
\usepackage{amsmath,amssymb,mathtools}
\usepackage{lipsum}
\usepackage{algorithm}
%\usepackage{algorithmic}
\usepackage[noend]{algpseudocode}
\usepackage{tikz-cd}
\usetikzlibrary{decorations.pathmorphing}
\usepackage{subcaption}
\usepackage{hyperref}
\usepackage{cite}
\renewcommand{\algorithmiccomment}[1]{$\triangleright$ #1}
\usetikzlibrary{matrix}
 %\usepackage[demo]{graphicx}
% \usepackage{caption}


\linespread{1.5} 

\usepackage{xepersian}
\settextfont{XBZar}
\setdigitfont{XBZar}

\title{امضای دیجیتال مقاوم کوانتومی بر اساس همسانی های بین خم های سوپرسینگولار}
\author{مصطفی قربانی
	\\[1cm]{ استاد راهنما: دکتر حسن دقیق}}
%\author{مصطفی قربانی}
\date{}

\begin{document}
\maketitle

توجه به این نکته لازم است که تساوی زیر برقرار است :
$$ (E /  \langle S \rangle) /  \langle \phi(R) \rangle = 
E /  \langle R,S \rangle = 
(E/ \langle R \rangle) /  \langle \psi(S) \rangle	
$$
\\














\iffalse
\textbf{توجه.}
برای دست‌یابی به 
$\lambda$
بیت امنیت ، لازم است که عدد اول 
$p$
انتخابی ،‌ حتما
$6\lambda$
بیت باشد و پروتکل بالا حتما 
$\lambda$
بار تکرار شود. اگر ویکتور تمام 
$\lambda$
مرحله از پروتکل را تایید کند ، آنگاه اثبات هویت پگی مورد قبول قرار می‌گیرد(ادعای او مبنی بر دانش کلید خصوصی
$S$
اثبات می‌شود) و در غیر اینصورت ویکتور متقاعد نمی‌شود و آن را رد می‌کند. 
~
\\
\fi






\\

















\end{document}