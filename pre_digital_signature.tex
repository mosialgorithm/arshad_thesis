
\chapter{امضای‌دیجیتال}\LTRfootnote{Digital Signature}\label{DS}
فرض کنید شخصی به‌نام پوریا خواهان آن باشد تا سندی(دیجیتالی) را (در محیطی ناامن مانند اینترنت) منتشر کند با این ویژگی که سند گویای آن باشد که از طرف پوریا می‌باشد و درنتیجه موردتایید وی نیز می‌باشد. بدین منظور پوریا باید از پروتکلی به نام امضای دیجیتال که نوعی رمزنگاری نامتقارن می‌باشد، استفاده کند. 
\\
در امضای دیجیتال ، شخصی می‌تواند پیام خود را امضا و آن را منتشر کند و هر شخص دیگری با دیدن پیام و امضا  پی ببرد که آیا پیام دریافت شده از شخص موردنظر می‌باشد یا خیر. بدین منظور لازم است مکانیزمی وجود داشته باشد تا شخص گیرنده پیام ، صحت هویت فرستنده را تایید کند. برای اجرای این سیستم پوریا به عنوان امضاکننده می‌تواند از یک کلیدخصوصی برای امضای پیام و از یک کلیدعمومی متناسب با کلید خصوصی (کلید خصوصی و کلید‌عمومی با هم وابسته و مربوط هستند و توسط الگوریتمی مشخص تولید می‌شوند) برای تایید امضا استفاده کند. 
\\
در ادامه می‌خواهیم نشان دهیم در بطن این پروتکل، دو پروتکل دیگر به نام  احراز‌هویت  و اثبات دانش صفر نیز اجرا می‌شوند. به عبارت دیگر با امضای سند، امضاکننده هویت خود را نشان داده است و هر شخصی که سندامضاشده را ببیند متوجه می‌شود که پوریا آن را امضا کرده است و بنابراین پروتکل احرازهویت صورت گرفته است. در طرف دیگر پروتکل اثبات دانش صفر نیز انجام شده است به این صورت که بدون آن‌که دانش امضاکننده یعنی کلیدخصوصی افشا شود، ویکتور(تاییدکننده) توانسته است امضا را با استفاده از کلیدعمومی(تاییدساز) که توسط امضاکننده همراه با پیام  منتشر شده بود تاییدکند. 
\\
در ادامه می‌خواهیم پروتکل امضا را براساس پروتکلی به‌نام اثبات‌دانش‌صفر‌هویت پیاده‌سازی کنیم.

\section{ طرح‌های تعهد}\label{commitment_schemes}


برای نشان‌دادن ایده این پروتکل، این سوال را مطرح می‌کینم  که چگونه می‌توان از طریق ایمیل به بازی سنگ، کاغذ، قیچی پرداخت؟ 
\\
اگر بخواهیم به روش مرسوم بازی کنیم و انتخاب‌های خود را با ایمیل ارسال کنیم، شخص باب می‌تواند پس از دریافت انتخاب آلیس از طریق ایمیل، انتخاب خود را برای به‌دست‌اوردن پیروزی تغییر دهد و انتخابی را از طریق ایمیل ارسال کند که با آگاهی از انتخاب طرف دیگر بازی  به‌دست آورده است! و لذا انجام بازی  به‌صورت  عادلانه میسر نیست.
\\
همچنین لازم به ذکر است که احتمال اینکه هردو انتخاب واقعا در یک زمان به طرف مقابل ارسال شود غیرممکن است. حتی واردکردن فرد سوم به بازی به‌عنوان داور هم نمی‌تواند کمکی به اجرای عادلانه بازی داشته باشد زیرا ممکن است یکی از طرف‌های بازی با داور تبانی کند و از حرکت طرف مقابل آگاه شود.
\\
 بنابراین براساس گفته‌های بالا  به این نتیجه می‌رسیم که عملا این بازی از طریق ایمیل با روش معمولی امکان‌پذیر نخواهد بود.
اما راهی وجود دارد که حتی اگر آلیس شروع‌کننده‌ی بازی باشد و باب پاسخ آن را از طریق ایمیل دریافت کند با این حال باب انتخابی که انجام می‌دهد واقعا تصادفی است و از انتخاب آلیس هیچ‌گونه اطلاعی ندارد. این راه‌حل از طریق طرح تعهد قابل پیاده‌سازی می‌باشد که در ادامه به معرفی آن می‌پردازیم.

\subsection{معرفی}\label{commitment_definition}
یک طرح تعهد شامل دو پروتکل به‌نام‌های تعهد و آشکارسازی می‌باشد که معمولا بین دو بخش که به فرستنده و دریافت‌کننده شناخته می‌شوند شکل می‌گیرد. در بیشتر حالت‌ها پروتکل‌های تعهد و آشکارسازی در یک الگوریتم تجمیع ‌می‌شوند. همچنین لازم به ذکر است که برای ارتباط بین فرستنده و دریافت‌کننده نیاز به هیچ تعامل دو سویه نمی‌باشد و درنتیجه این طرح اساسا یک طرح غیرتعاملی می‌باشد.

\begin{definition}\label{commit_scheme_define}
اگر  تعهد یک الگوریتم با زمان چندجمله‌ای و  معین به صورت زیر باشد:
$$ commit : \{ 0,1\}^ k \times \{ 0,1\}^ * \rightarrow \{ 0,1 \}^* $$
که
$k$
پارامتر امنیتی می‌باشد، آنگاه یک طرح تعهد(غیرتعاملی) شامل دو پروتکل بین فرستنده و دریافت‌کننده  می‌باشد که به صورت زیر بیان می‌شوند:
\begin{itemize}
\item[]{
\textbf{مرحله‌ی تعهد .}
در این پروتکل، فرستنده مقدار انتخابی خود یعنی 
$x \in \{ 0,1 \}^*$
را معین و تابع
$C = commit(u,x)$
را محاسبه می‌کند تا مقدار انتخابی خود یعنی 
$x$
با مقدار تصادفی اما معین 
$ u ~ {\in}_R ~ \{ 0,1\}^k$
تلفیق شود و درنتیجه انتخابش، مقداری تصادفی به خود بگیرد. در این مرحله، فرستنده به‌جای مقدار 
$x$
، مقدار 
$C$
 را به‌عنوان تعهدی بر انتخاب اصلی خود ارسال می‌کند. و در طرف دیگر دریافت‌کننده مقدار
 $C$
 را برای ادامه طرح، دخیره می‌کند.
\\
درمثال بالا می‌توان گفت،
$x$
بیانگر حرکت و انتخاب هر بازیکن می‌باشد و 
$u$
یک مقدار تصادفی است که هیچ ارتباطی با انتخاب
$x$
آن ندارد. 
$C$
تعهد و مقداری است که انتخاب 
$x$
در آن مخفی شده است و از طریق ایمیل برای طرف مقابل بازی ارسال می‌شود.
}

\item[]{
\textbf{مرحله‌ی آشکارسازی .}
پروتکلی است که در طی آن فرستنده با ارسال
$u$
و
$x$
برای دریافت‌کننده،  با کمک تابع 
$C = commit ~ (u,x)$
توانایی آشکارسازی مقدار
$x$
از تعهد
$C$ 
را به دریافت‌کننده می‌سپارد.
به‌عبارت‌دیگر در این مرحله دریافت‌کننده به محاسبه‌ی تابع
$commit(u,x)$
می‌پردازد و بررسی می‌کند که آیا خروجی این تابع با مقدار  
$C$
که قبلا دریافت کرده است برابر هستند یا خیر.
\\
برای شبیه‌سازی این مرحله در مثال بالا می‌توان گفت که پس از ارسال تعهدها از طرف آلیس و باب برای یکدیگر در مرحله‌ی قبل، در این مرحله بازی هریک از بازیکن‌ها به بررسی تعهد طرف مقابل پرداخته و درانتها با نمایش انتخاب‌ها پیروز بازی مشخص می‌شود.
}
\end{itemize}

\end{definition}~

لازم به ذکر است که یک حالت خاص از طرح بالا زمانی است که مقدار تعهد شده تک بیتی یعنی 
$x \in \{0,1\}$
باشد که به 
\textbf{طرح تعهد بیتی}
 معروف است. به‌عبارت دیگر برای انجام این پروتکل  فرستنده تنها دو انتخاب دارد که ارزش آن یا صفر یا یک می‌باشد.
 \\
 \\
 برای امنیت کامل این طرح لازم است تا ویژگی‌های زیر در هر طرح تعهدی وجود داشته باشد:
 
\begin{itemize}
\item []{
\textbf{انقیاد .}\LTRfootnote{Binding}
فرستنده نباید بعد از ارسال تعهد
$C$
 قادر به تغییر مقدار 
$x$
 شود.  
 در مثال بالا می‌توان گفت که  آلیس بعد از ارسال حرکت خود از طریق ایمیل، نمی‌تواند بعد از انتخاب حرکت باب، انتخاب خود را تغییر دهد. 
}
	
	
\item[]{
\textbf{‌مخفی‌سازی .}\LTRfootnote{Hiding}
بیانگر آن است که هنگام دریافت تعهد توسط دریافت‌کننده، مقدار موردنظر 
$x$
از طریق تعهد
$C$
قابلیت کشف نداشته باشد.
در مثال بالا می‌توان گفت باب با دریافت تعهد آلیس از طریق ایمیل قابلیت کشف مقدار انتخاب‌ شده‌ی آلیس را ندارد و انتخاب خود را کاملا تصادفی انتخاب می‌کند.
}
\end{itemize}

اگر بخواهیم ویژگی‌های بالا را به‌صورت یک تابع ریاضی پیاده‌سازی کنیم، این دو ویژگی به صورت زیر خواهند بود:
\begin{itemize}
	
\item[]{
\textbf{مقاومت ثانویه .}
برای هر ورودی داده شده، ورودی دیگری که خروجی مشابهی را موجب شود، دشوار باشد. به عبارت دیگر برای هر متخاصم
$\varepsilon$
احتمال تولید
$u,u' \in \{0,1\}^k$
که رابطه‌ی 
$commit(u,0) = commit(u',1)$
را موجب شود، ناچیز باشد.
}
	
	
\item[]{
\textbf{معکوس‌سازی سخت (یکطرفه )}
با داشتن ورودی محاسبه خروجی آسان است اما از طریق خروجی محاسبه ورودی آن مشکل است.
 به‌عبارت دیگر می‌توان گفت توزیع‌های ناشی از
 $commit(u,0)$
 و
 $commit(u,1)$
 زمانیکه
 $u~ {\in}_R \{0,1\}^k $
 ،غیرقابل‌تشخیص(یکنواخت) باشند.
}
\end{itemize}~
\iffalse
علاوه‌براین، یک طرح تعهد، انقیادمحاسباتی
\LTRfootnote{computationally binding}
 خوانده می‌شود اگر متخاصم
$\varepsilon$
محدود به یک الگوریتم احتمالاتی چندجمله‌ای باشد. به‌عبارت دیگر چون متخاصم محدودیت محاسباتی دارد بنابراین طرح ما از امنیت برخوردار است. درغیراینصورت یعنی زمانیکه متخاصم هیچ‌گونه محدودیت محاسباتی نداشته باشد آنگاه این طرح را انقیاد اطلاعاتی نظری
\LTRfootnote{information-theoretically binding}
می‌نامیم و به بیان بهتر می‌توان گفت که از نظر نظری امنیت سیستم پابرجاست ولی درعمل با  این فرض که متخاصم هیچ محدودیت محاسباتی ندارد امنیت طرح لزوما برآورده نمی‌شود.
\\
\\
به‌طور مشابه اگر توزیع‌های ناشی از 
$commit(u,0)$
و
$commit(u,1)$
 نیز به‌صورت محاسباتی غیرقابل‌تشخیص باشند آنگاه طرح تعهد را مخفی‌سازی محاسباتی
 \LTRfootnote{computationally hiding}
 گوییم و همچنین طرح را مخفی‌سازی اطلاعاتی نظری
 \LTRfootnote{information-theoretically hiding}
 گوییم اگر این توزیع‌ها از نظر آماری غیرقابل‌تشخیص باشند.
\\
\\
در پایان ذکر این نکته نیز لازم است که امنیت بالا تنها در برابر حملات فرستنده یا گیرنده صدق می‌کند. به‌طورمثال اگر فرض شود بخش
$\mathcal{A}$
به‌عنوان فرستنده و بخش
$\mathcal{B}$
در نقش دریافت‌کننده باشد و تعهد
$C$
از طرف
$\mathcal{A}$
برای
$\mathcal{B}$
ارسال می‌شود آنگاه هیچ ضمانتی وجود ندارد که دریافت‌کننده
$\mathcal{B}$
متوجه شود که یک متخاصم مقدار تعهد
$C$
را با تعهدجعلی
$C' = commit(u',x')$
را درحین اجرای پروتکل تعهد جایگزین کرده است و درنتیجه مقدارهای
$u,x$
را با مقدارهای
$u',x'$
را در حین پروتکل آشکارسازی جایگزین کرده است. برای جلوگیری از این حمله استفاده از یک کانال احراز شده بین
$\mathcal{A}$
و
$\mathcal{B}$
ضروری می‌باشد.
\fi

\example{
	اگر یک تابع هش رمزنگاری
	$H$
	را دراختیار داشته باشیم آنگاه طرح تعهد بیتی خود را می‌توانیم به صورت زیر به‌دست بیاوریم:
	$$ {commit}_0(u,x) = H(u,x) $$
	که درآن مقدار تعهد
	$x \in \{0,1\}$
	و
	$u \in \{0,1\}^k $
	می‌باشند.
	
	\begin{itemize}
		\item {
			ویژگی مقاوم-تصادم تابع هش
			$H$
			ضمانت می کند که شخص متعهدشده نمی‌تواند 
			$u$
			،
			$x$
			و
			$u'$
			،
			$1-x$
			را به‌دست آورد بطوریکه
			$$ H(u,x) = H(u',1-x) $$
			بنابراین طرح ما خاصیت انقیاد را دارا می‌باشد.
		}
		
		\item {
			ویژگی مقاوم در برابر تضاهر برای خاصیت مخفی‌سازی لازم می‌باشد ولی با این ویژگی هیچ ضمانتی نیست که مقدار
			$x$
			مخفی(به اندازه مقدار
			$u$
			) بماند.بدین منظور لازم است که از مدل اوراکل تصادفی استفاده کنیم.
		}
	\end{itemize}

}

\subsection{طرح تعهد در اثبات دانش صفر}
طرح تعهد در بسیاری از پروتکل‌های رمزنگاری مورداستفاده قرار می‌گیرند. از جمله‌ی این پروتکل‌ها می‌توان به اثبات دانش صفر اشاره کرد.
در ادامه به‌ دلایل استفاده اثبات دانش صفر از طرح تعهد می‌پردازیم.



\section{اثبات دانش صفر هویت}\label{ZKPOI}

به‌طور رسمی، یک سیستم اثبات دانش صفر یک رویه است که طی آن پگی(به‌عنوان شخص اثبات‌کننده)، ویکتور(به‌عنوان شخص تاییدکننده) را متقاعد می‌کند که به یک حقیقت معین  
\RTLfootnote{
حقیقت می‌تواند هویت اثبات‌کننده (پگی) ‌باشد.
}
اشراف دارد بطوریکه هیچ اطلاعات اضافی نسبت به دانش خود در اختیار ویکتور قرار نمی‌دهد تا بدین‌منظور خود ویکتور نتواند به عنوان یک مدعی، دیگران را متقاعد کند که به حقیقت مورد بحث اشراف دارد. برای توضیح بیشتر این پروتکل مثالی ارائه می‌کنیم.
\begin{example}\label{ex_zkp}\textbf{اثبات دانش صفر}
	
فرض کنید دو لیوان شفاف در اختیار داریم که یکی حاوی آبِ خالص و دیگری حاوی آب با مخلوطی شفاف می‌باشد که فقط پگی فرق این دو لیوان را می‌داند. حال برای آن‌که پگی به ویکتور ثابت کند که دانش لازم را برای تشخیص لیوان حاوی آب خالص و لیوان آب ناخالص را در اختیار دارد می‌بایست به چالش‌هایی که از طرف ویکتور مورد سوال قرار می‌گیرد به درستی جواب بدهد. ویکتور برای اطمینان از اینکه پگی واقعا دانش لازم این اثبات را می‌داند می‌تواند چالش‌های خود را چندین بار تکرار کند و اگر پگی در تمامی چالش‌ها به درستی جواب بدهد آنگاه مطمئن می‌شود که پگی دانش لازم را دراختیار دارد. پگی برای آن‌که مستقیما دانش خود را افشا نکند لیوان حاوی آب خالص را به ویکتور نشان نمی‌دهد و در عوض مراحل زیر به تعداد مشخصی تکرار می‌شود
\begin{enumerate}
	\item 
	ابتدا پگی به عنوان اثبات‌کننده ادعا می‌کند که مکان لیوان‌ حاوی آب ناخالص  را می‌داند.
	\item 
 پگی چشمان خود را با چشم‌بند می‌بندد و سپس	ویکتور به‌عنوان یک  چالش ، یا جای دو ظرف آب را باهم جا‌به‌جا می‌کند یا بدون تغییر آن‌ها آماده پاسخ چالش خود می‌شود
	\item 
	پگی چشم‌بند را از چشمان خود برمی‌دارد و با اتکا به دانشی که در اختیار دارد مشخص می‌کند که آیا جای این دو لیوان عوض شده است یا خیر
	\item 
	اگر پگی به درستی تشخیص دهد که جای لیوان‌ها عوض شده است یا خیر  آنگاه ویکتور برای اطمینان از شانسی نبودن جواب پگی می‌تواند بار دیگر مراحل را با همکاری پگی تکرارکند اما اگر حتی یک بار پگی به اشتباه جواب چالش پگی را بدهد آنگاه ویکتور با اطمینان ادعای پگی را نمی‌پذیرد.
\end{enumerate}~
\\
ذکر این نکته لازم است که اگر پگی به صورت شانسی به چالش جواب بدهد به احتمال
$1/2$
به طور صحیح  جواب داده است ، حال اگر  که رویه اثبات به تعداد 
$n$
بار تکرار شود آنگاه به احتمال 
$1 / 2^n$
 پگی به صورت شانسی جواب چالش‌ها را  به درستی داده است ، همان‌گونه که مشخص است تقریبا محال است که همچین اتفاقی رخ دهد و پگی قادر باشد که تمام جواب‌ها را به صورت شانسی جواب داده باشد. بنابریان ویکتور با تکرار رویه اثبات و جواب صحیح پگی در هر مرحله، کاملا قانع می‌شود که پگی به دانش ادعا شده اشراف دارد.

\end{example}~
\\
برای پیاده‌سازی پروتکل‌اثبات‌دانش‌صفر به صورت ریاضی، از همسانی بین  خم‌های سوپرسینگولار استفاده می‌کنیم.  
\\
در طرحی که خواهان ارائه آن هستیم به خم‌های سوپرسینگولار با درجه‌ای هموار
\RTLfootnote{
در مبحث خم‌های سوپرسینگولار، ساخت خم‌های با درجه هموار آسان می‌باشد و با استفاده از این خم‌ها می‌توان تعداد زیادی همسانی بین آنها ساخت که خیلی سریع قابل محاسبه هستند.
}
 نیاز می‌باشد بنابراین ابتدا  عدد اولی به فرم 
$p = \ell_A^{e_A} \ell_B^{e_B} \cdot f \pm 1$
را انتخاب می‌کنیم که 
$\ell_A$
و
$\ell_B$
اعداد اول کوچک (معمولا ۲ و ۳) می‌باشند با این خاصیت که طول ارقام
$\ell_A^{e_A}$
و
$\ell_B^{e_B} $
برابر باشد (در بخش ۶ به طور مفصل آن را بررسی خواهیم کرد) و همچنین  
$f$ % section 6 dont defind until now !!!!!!!!!!!!!!!!!!!!!!!!
یک عامل کوچک است که باعث می‌شود 
$p$
یک عدد اول شود(از آنجا که 
$\ell_A^{e_A}$
و
$\ell_B^{e_B} $
دیگر اول نیستند). در ادامه ‌با روش بروکر 
\cite{broker}
، یک خم سوپرسینگولار 
$E$
را روی میدان 
$F_{p^2}$
با مرتبه‌ی 
$(\ell_A^{e_A} \ell_B^{e_B})^2$
 به دست می‌آوریم. سپس دو زیرگروه 
$E[\ell_A^{e_A}]$
و
$E[\ell_B^{e_B}]$ 
که مولدهای آن به ترتیب به صورت زوج نقاط
$ \langle P_A,Q_A \rangle $
و
$ \langle P_B,Q_B \rangle  $
می‌باشند را روی خم 
$E$
محاسبه می‌کنیم.
\\
طرح اثبات دانش صفر براساس همسانی‌ها مطابق شکل 
\ref{fig:zkp}
 صورت می‌گیرد. پگی به عنوان اثبات کننده، نقطه 
$S$
که تولیدکننده هسته همسانی 
$\phi : E \longrightarrow E/ \langle S \rangle $
می‌باشد را به عنوان دانش  خود، به صورت مخفی نزد خود نگه  می‌دارد. 
و بر این اساس، کلید خصوصی و کلیدعمومی پگی به‌صورت زیر معرفی می‌شود :

\begin{itemize}
\item {\textbf{کلید خصوصی :} }
هسته همسانی
$\phi$
یعنی
$S$

\item {\textbf{کلید عمومی :} }
خم بیضوی
$E/ \langle S \rangle$
و تصویر نقاط  
$P_B$ 
و
$Q_B$
یعنی
$\phi(P_B)$
و	
$\phi(Q_B)$	
\end{itemize}~ 
% ==============================================================================================
% figure 1 for demonstrate zer-konwledge proof of identity
\begin{figure}[H] 
	\begin{center}
		
		\begin{tikzcd}
			E \arrow[r, "\phi"] \arrow[d, "\psi"] & E/ \langle S \rangle \arrow[d, "{\psi}' "] \\
			E/ \langle R \rangle \arrow[r, "{\phi}' "] & E/\langle R,S \rangle
		\end{tikzcd}
		
		\caption{
			هر فلش با همسانی و و هسته‌اش نشانه گذاری شده است    
		}
		\label{fig:zkp}
		
	\end{center}
\end{figure}
% ==============================================================================================
\vskip 0.5in
حال پگی برای آن که به ویکتور (تاییدکننده) ثابت کند که دانش 
$\langle S \rangle $
را می‌داند ، مراحل زیر به ترتیب انجام می‌شود:
\begin{enumerate}
	
\item {
\begin{itemize}
\item 
 نقطه تصادفی 
$R$
را از مرتبه‌ی
$\ell_B^{e_B}$
انتخاب می‌کند.
	
\item 
 همسانی
$\psi : E \rightarrow E / \langle R \rangle$
را محاسبه می‌کند.
\item 
در ادامه همسانی‌های
${\phi}' : E / \langle R \rangle \rightarrow  E / \langle R,S \rangle $
را باهسته
$\langle \psi(S) \rangle$
و همچنین همسانی
${\psi}' : E / \langle S \rangle \rightarrow  E / \langle R,S \rangle $
را با هسته‌ی
$\langle \phi(R) \rangle$
 از طریق فرمول ولو محاسبه می‌کند.
 \item 
 پس از محاسبات بالا ، پگی تعهد 
 \LTRfootnote{commitment}
 $com = (E_1 , E_2)$
را برای ویکتور ارسال می‌کند که
 $E_1 = E / \langle R \rangle$
 و
 $E_2 = E / \langle R,S \rangle$
‌ می‌باشند.
 
\end{itemize}	
} % end item

\item 
ویکتور به طور تصادفی  بیت چالشی 
$ch \in \{0,1\}$
را انتحاب و برای پگی ارسال می‌کند.
\item 
پگی پاسخ 
$resp$
را برای ویکتور ارسال می‌کند :
\begin{itemize}
	\item
	اگر 
	$ch = 0$
	آنگاه
	$resp = (R,\phi(R))$
	
	\item
	اگر 
	$ch = 1$
	آنگاه
	$resp = \psi(S)$
	
\end{itemize}
\item {
\begin{itemize}
	\item 
	اگر
	$ch = 0 $
	، ویکتور ابتدا بررسی می‌کند که آیا
	$R$
	و
	$\phi(R)$
	هردو از مرتبه‌ی 
	$\ell_B^{e_B}$
	هستند یا خیر . در ادامه بررسی می‌کند که آیا این دو، هسته‌ی همسانی‌های
	$E \rightarrow E_1$
	و 
	$E/ \langle S \rangle \rightarrow E_2$
	را تولید می‌کنند یا خیر.
	\item 
	اگر 
	$ch = 1 $
	، ویکتور ابتدا بررسی می‌کند که آیا  
	$\psi(S)$
	از مرتبه‌ی
	$\ell_A^{e_A}$
	می‌باشد یا خیر و همچنین درادامه بررسی می‌کند که آیا هسته‌ی همسانی 
	$E_1 \rightarrow E_2$
	را تولید می‌کند یا خیر.
\end{itemize}
} % end item
\end{enumerate}~
\remark
قابل ذکر است که در مرحله‌ی ۴، اگر نتیجه‌ی بررسی تاییدکننده مثبت باشد، ادعای اثبات‌کننده تایید و در غیر اینصورت ادعای وی رد می‌شود.	
\\
\remark
همچنین قابل ذکر است که رابطه‌ی زیر همواره برقرار است:
$$
\frac{(E/\langle S \rangle)}{\langle \phi(R) \rangle} =
\frac{E}{\langle R,S \rangle}  = 
\frac{( E/ \langle R \rangle)}{\langle \psi(S) \rangle} 
$$


% ==============================================================================================
% fugure 2 for demostrate DSSP assumption for chaaleng ????
\begin{figure}[H]
\centering
\begin{subfigure}[b]{0.4\linewidth}
\caption{b = 0}
\label{fig:sub1}
\begin{tikzcd}
	E \arrow[r, dashrightarrow , "\phi"] \arrow[d, "\psi"] & E/ \langle S \rangle \arrow[d, "{\psi}' "] \\
	E/ \langle R \rangle \arrow[r, dashrightarrow ,"{\phi}' "] & E/\langle R,S \rangle
\end{tikzcd}

\end{subfigure}


\begin{subfigure}[b]{0.4\linewidth}
\caption{b = 1}
\label{fig:sub2}
\begin{tikzcd}
	E \arrow[r, dashrightarrow , "\phi"] \arrow[d, dashrightarrow , "\psi"] & E/ \langle S \rangle \arrow[d, dashrightarrow , "{\psi}' "] \\
	E/ \langle R \rangle \arrow[r, "{\phi}' "] & E/\langle R,S \rangle
\end{tikzcd}
	
\end{subfigure}%

	% -------------------------------------------------------------------
	\caption{
			همسانی‌های مخفی با خط‌های مقطع نمایش داده شده است. خط های توپر نمایش دهنده همسانی‌هایی می‌باشد که پگی نسبت به چالش انجام شده ظاهر می‌کند. با این حال همسانی‌های ظاهرشده هیچ اطلاعاتی درباره همسانی مخفی
			$\phi$
			افشا نمی‌کند.	
	}
	\label{fig:challenge}
\end{figure}~
% ==============================================================================================

 برای دست‌یابی به 
$\lambda$
بیت امنیت، لازم است که عدد اول
$p$
دقیقا 
$6\lambda$
بیت باشد(دلیل این امر در بخش ۶ ذکر شده است) و پروتکل 
$\lambda$
بار تکرار شود. اگر ویکتور(تاییدکننده) تمام
$\lambda$
بار تکرار پروتکل را با موفقیت تایید کند آنگاه پگی(اثبات‌کننده) توانسته است ادعای خود مبنی بر دانش کلیدخصوصی
$S$
را برای ویکتور اثبات کند، در غیراینصورت ادعا توسط ویکتور موردقبول قرار نمی‌گیرد.


لازم به ذکر است که در هربار اجرای این طرح، نقطه
$R$
کاملا به صورت تصادفی انتخاب می‌شود و براساس این نقطه، شکل
\ref{fig:zkp}
حاصل می‌شود و بنابراین همسانی‌ها و خم‌های هر بار اجرای این پروتکل با دفعه قبل کاملا فرق دارد و فقط خم 
$E / \langle S \rangle $
ثابت باقی می‌ماند.
\\
 همچنین چنان‌‌که در بخش 
6
خواهیم دید اگر همسانی‌های 
$\psi$
و
${\psi}'$
و
${\phi}'$
همزمان معلوم شوند، دانش  
$S$
 قابل کشف است بنابراین برای جلوگیری از افشای 
$S$
، در هر بار تکرار رویه اثبات، درمقابل چالش 
 $ch = 0 $
 همسانی‌های 
 $\psi$
 و
 ${\phi}'$ 
 عمومی می‌شوند و درمقابل چالش
 $ch = 1$
 تنها همسانی 
 ${\phi}'$ 
 عمومی می‌شود تا طرح ما ایمن باقی بماند.
 


\section{پروتکل‌های شناسایی}\label{identification}
پروتکل‌های شناسایی
\LTRfootnote{Identification Protocols}
بین دو بخش به‌نام‌های تاییدکنننده و اثبات‌کننده رخ می‌دهد که در طی آن اثبات‌کننده خواهان متقاعدکردن تاییدکننده برای ادعای خود می‌باشد. یک مثال معمولی زمانی است که کاربری خواهان دسترسی به حساب کامپیوتری خود می‌باشد(ورود امن).
\\
عموما پروتکل‌های شناسایی ممکن است براساس یک یا بیشتر فاکتورهای زیر بناشده باشد:
\begin{itemize}
\item {
\textbf{آنچه شما هستید.}
 مانند اثرانگشت، اسکن چشم و ..
}
\item {
\textbf{آنچه شما در اختیار دارید.}
مانند کارت‌هوشمند، سیم‌کارت یا هر توکن‌های سخت‌افزاری 
}

\item {
\textbf{آنچه شما می‌دانید. }
مانند گذرواژه، کلیدهای مخفی و ..
}
\end{itemize}
درادامه خواهان آن هستیم تا پروتکل‌های شناسایی رمزنگاری را که درآن یک اثبات‌کننده تنها نیاز به داشتن( دانستن) یک کلیدمخفی است را بیان کنیم. ساختارهای رمزنگاری بسیار زیادی برای پروتکل‌های شناسایی موجود می‌باشد  و هدف عمومی این است که مقدار محاسبات برای بخش‌های تاییدکننده و همینطور اثبات‌کننده کاهش یابد.
\iffalse
حدوده‌ی امنیت از نسبتا ضعیف پروتکل‌های گذرواژه مبنا تا قوی برای پروتکل‌های اثبات‌دانش‌صفر گسترده می‌باشد.

\\
لازم به ذکر است که مشکلات مربوط به پروتکل‌های شناسایی، ویژه‌ی احراهویت متن (امضاهای دیجیتال) و همچنین مبادله کلید تاییدشده می‌باشد.
\fi

\subsection{معرفی}\label{identification_define}

یک پروتکل شناسایی دراصل قسمتی از یک طرح شناسایی می‌باشد.
یک طرح  کامل شناسایی شامل دو پروتکل به‌نام‌های ثبت‌نام و شناسایی است که میان دو بخش به‌نام‌های اثبات‌کننده و تاییدکننده صورت می‌پذیرد.
طرح‌های شناسایی به دو صورت متقارن و نامتقارن پیاده‌سازی می‌شوند.
دریک طرح شناسایی متقارن ، با اشتراک یک کلیدخصوصی بین هردو بخش، پروتکل ثبت‌نام پایان می‌یابد.
در یک طرح شناسایی نامتقارن ، پروتکل  ثبت‌نام پس از اشتراک یک کلیدعمومی بین  تمام بخش‌ها پایان می‌یابد و تنها اثبات‌کننده از کلیدخصوصی مطلع است.(البته در طرح‌های پیشرفته،  ممکن است تاییدکننده نیز دارای کلیدخصوصی باشد)
\\
یکی از مزیت‌های بزرگ طرح نامتقارن این است که اثبات‌کننده ممکن است چندین مرتبه از کلیدعمومی‌اش در ارتباط با تاییدکننده استفاده کند.
\\
از انواع پروتکل‌های ارائه‌شده برای طرح شناسایی، می‌توان به 
\textbf{طرح گذرواژه مبنا}
، 
\textbf{زنجیره‌ی هش یکطرفه}
 و 
\textbf{پروتکل‌های چالش-پاسخ} 
  اشاره کرد.
از آنجا که در طرح‌امضای ارائه شده در این پایان‌نامه از پروتکل‌های چالش-پاسخ استفاده شده است لذا در ادامه به معرفی این پروتکل خواهیم پرداخت.

\subsection{امنیت طرح شناسایی }\label{security_of_identification}
در این قسمت به حمله‌هایی که ممکن است به طرح شناسایی صورت پذیرد را بررسی می‌کنیم.
با درنظرگرفتن اینکه پروتکل ثبت‌نام در یک محیط امن انجام می‌شود درنتیجه تنها به بررسی حملات رمزنگاری به پروتکل شناسایی می‌پردازیم.
اساسی‌ترین پیش‌نیاز امنیت برای پروتکل شناسایی، متوقف کردن حملات جعل‌هویت
\LTRfootnote{impersonation attacks}
می‌باشد، که در این صورت برای یک مهاجم غیرممکن است تا هویت خودش را  (در برابر  با تاییدکننده یا اثبات‌کننده)با موفقیت جعل کند. در ادامه به معرفی چندین حمله‌ی جعل‌هویت از نوع فعالانه و غیرفعالانه می‌پردازیم.


از اصلی‌ترین حملات جعل‌هویت غیرفعال می‌توان به استراق‌سمع روی یک ارتباط بین اثبات‌کننده و تاییدکننده در یک اجرای قانونی پروتکل شناسایی اشاره کرد. نوع دیگر حملات غیرفعال، حمله‌ی کلید در طرح غیرمتقارن می‌باشد، که درآن حمله‌کننده تلاش دارد تا از طریق کلیدعمومی به کلیدخصوصی دسترسی یابد.
\\
یک فرم ساده از حمله‌ی جعل‌هویت فعال، حمله‌ی حدس
\LTRfootnote{guessing attack}
می‌باشد که درآن حمله‌کننده در نقش اثبات‌کننده درآمده و امیدوار است تا بدون دانستن کلیدخصوصی یا کلیدمخفی اثبات‌کننده، حدس درستی درنظر گرفته باشد.میزان موفقیت حمله‌ی حدس می‌تواند با ترکیب با حمله‌ی تاییدکننده‌ی متقلب
\LTRfootnote{cheating verifier}
افزایش یابد.
در حمله‌ی تاییدکننده‌ی متقلب، متخاصم خود را درنقش تاییدکننده جا می‌زند و امیدوار است تا اطلاعات مفیدی را از طریق اثبات‌کننده با انحراف پروتکل استخراج کند.
\\
در نهایت متخاصم می‌تواند حمله‌ی مردمیانی 
\LTRfootnote{man-in-the-middle}
را اجرا کند. در این حمله، اثبات‌کننده صادق
$\mathcal{P}$
فکر می‌کند که پروتکل شناسایی را با تاییدکننده‌
$\mathcal{V}^*$
اجرا می‌کند، اما درواقع امر،
$\mathcal{V}^*$
تمام پیام‌های ردوبدل شده بین خود و تاییدکننده صادق
$\mathcal{P}$
را به یک تاییدکننده
$\mathcal{V}$
می‌فرستد که خود تاییدکننده
$\mathcal{V}$
نیز فکر می‌کند که در اصل پروتکل را با یک تاییدکننده
$\mathcal{P}$
اجرا می‌کند. به بیان ساده‌تر متخاصم بین تاییدکننده و اثبات‌کننده قرار می‌گیرد و تمام تعاملات از طریق کانال متخاصم انجام شده و هردو بخش تاییدکننده و اثبات‌کننده به این موضوع پی نمیبرند.
\\
حمله مردمیانی یادآور حمله‌ای در شطرنج به‌نام استادبزرگ می‌باشد. در این حمله، یک بازیکن آماتور شطرنج تلاش دارد با بازی کردن همزمان با دو استادشطرنج در یک زمان، مهارت و امتیاز خود را افزایش دهد. در این حمله، یازیکن آماتور دریک بازی مهره‌سفید و در بازی دیگر مهره‌سیاه می‌باشد و با شروع بازی با استادبزرگ اول با مهره مشکی و کپی کردن حرکات شطرنج بین این دواستادبزرگ درانتظار پایان یازی می‌ماند. در پایان بازی‌ها یا بازیکن آماتور در یک بازی پیروزشده و متعاقبا در بازی دیگر شکست خورده و یا هردو بازی به نتیجه مساوی خاتمه یافته است.در پایان با هر کدام از حالت‌های فوق امتیاز بازیکن آماتور  به‌طورقابل‌توجهی افزایش یافته است.

\subsection{پروتکل‌های پرسش-پاسخ اصلی  }\label{ch_resp_protocol}

چهارنوع پروتکل پرسش-پاسخ اصلی وجود دارد. درهریک از این پروتکل‌های شناسایی، تاییدکننده با ارسال یک چالش تصادفی برای تاییدکننده، پروتکل را آغاز می‌کند. درادامه اثبات‌کننده با ارسال یک پاسخ برای تاییدکننده به‌منظور بررسی آن واکنش نشان می‌دهد. طرح کلی این پروتکل در شکل ۱۱۱ به‌صورت خلاصه آمده است.
درادامه برای هر طرح، حملات استراق‌سمع و تاییدکننده‌متقلب را بررسی خواهیم کرد. 


\subsubsection{رمزنگاری متقارن }\label{symmetric_encrypt}

فرض کنید اثبات‌کننده و تاییدکننده  یک کلیدمتقارن
$K ~{\in}_R ~ \{0,1\}^k $
را بین خود به اشتراک گذاشته‌اند. اگر
$E_K$
نشان‌دهنده‌ی الگوریتم رمزگذاری با استفاده از کلید 
$K$
و
$D_K$
نشنان‌دهنده‌ی الگوریتم رمزگشایی مربوطه باشد آنگاه به‌سادگی روشن است که:
$$E_K,D_K : \{0,1\}^k \rightarrow \{0,1\}^k$$

اگر به شکل ۱٫۱٫آ توجه شود پیداست که پروتکل شناسایی با ارسال چالش
$c ~{\in}_R ~ \{0,1\}^k$
توسط تاییدکننده آغاز می‌شود و در ادامه اثبات‌کننده پاسخ
$r = E_K(c)$
را برای تاییدکننده ارسال می‌کند. و درنهایت تاییدکننده تساوی
$D_K(r) = c$
را بررسی می‌کند. اگر تساوی برقرار باشد، پاسخ اثبات‌کننده تایید و درنتیجه شناسایی با موفقیت انجام می‌پذیرد و درغیراینصورت پاسخ اثبات‌کننده رد و پروتکل با موفقیت اتمام نمی‌یابد.
\\



برای ایستادگی در برابر حمله‌ی استراق‌سمع در یک طرح رمزگذاری باید از حمله‌ی متن ساده‌ی شناخته شده 
\LTRfootnote{known-plaintext attack (KPA)}
جلوگیری کرد. و برای  ایستادگی در برابر حمله‌ی تاییدکننده‌متقلب در این طرح لازم است از حمله‌ی متن‌ساده‌ی انتخابی
\LTRfootnote{chosen-plaintext attack(CPA)}
 تطبیقی جلوگیری کرد.
\\
 لازم به ذکر است که ‌حمله‌ی متن ساده‌ی شناخته شده یک نوع مدل حمله است که درآن حمله‌کننده به زوج متن ساده و ورژن رمزگذاری‌شده‌ی آن  دسترسی  دارد  و درصدد است با استفاده از  این زوج‌ها به کشف اطلاعات مخفی همچون کلیدهای خصوصی بپردازد.و حمله‌ی پیام ساده‌ی انتخابی نیز اشاره به مدلی از حمله دارد که در آن حمله‌کننده خواهان آن است تا یک متن ساده‌ی تصادفی را انتخاب و آن را رمزگذاری کند و درنتیجه نسخه‌ی 
رمزگذاری شده‌ی آن پیام را به‌دست آورد.

\subsubsection{احرازهویت متقارن }\label{symmetric_authectication}

فرض کنید اثبات‌کننده و تاییدکننده یک کلیدمتقارن
$K~ {\in}_R ~ \{0,1\}^k$
را بین خود به اشتراک گذاشته‌اند. همچنین 
$H : \{0,1\}^* \rightarrow \{0,1\}^k$
معرف یک تابع هش رمزنگاری می‌باشد.
\\
با توجه به شکل ۱.۱.ب 
پروتکل شناسایی با ارسال چالش
$c ~ {\in}_R ~ \{0,1\}^k$
توسط تاییدکننده برای اثبات‌کننده آغاز می‌شود.و متناسب با آن اثبات‌کننده نیز پاسخ
$r = H(K,c)$
را برای تاییدکننده ارسال می‌کند. درنهایت تاییدکننده رابطه‌ی
$r = H(K,c)$
را بررسی می‌کند.
\\
\\
در مدل‌های اوراکل تصادفی، این طرح در برابر حملات استراق‌سمع و تاییدکننده‌متقلب مقاوم می‌باشد. 

\subsubsection{رمزگذاری نامتقارن }\label{asymmetric_enccryption}

فرض کنید اثبات‌کننده و تاییدکننده یک کلیدعمومی
$pk$
را بین خود به‌اشتراک گذاشته‌اند که تنها اثبات‌کننده از کلیدخصوصی 
$sk$
متناظر با کلیدعمومی مطلع است.همنچنین 
$E_{pk}$
معرف یک الگوریتم رمزگذاری بااستفاده از کلیدعمومی
$pk$
می‌باشد و 
$D_{sk}$
نشان‌دهنده‌ی الگوریتم رمزگشای مرتبط با الگوریتم رمزگذاری است که ازکلیدخصوصی
$sk$
استفاده می‌کند.
\\
چنانکه در شکل ج مشاهده می‌شود ابتدا تاییدکننده پیام
$M ~ {\in}_R ~ \{0,1\}^k$
را انتخاب کرده و سپس با استفاده از الگوریتم رمزگذار، چالش
$c = E_{pk}(M)$
را برای اثبات‌کننده ارسال می‌کند و در ادامه اثبات‌کننده پاسخ موردانتظار
$r = D_{sk}(c)$
را تولید می‌کند. در انتها تاییدکننده تساوی
$r = M$
را بررسی می‌کند.
\\
\\
برای مقاومت در برابر حمله‌ی استراق‌سمع لازم است طرح رمزگذار ایمن معنایی باشد. و همچینین برای مقاومت دربرابر حمله‌ی تاییدکننده‌متقلب لازم است که این طرح در برابر حمله‌ی متن‌ساده‌انتخابی انطباقی ایمن باشد.

\subsubsection{احرازهویت نامتقارن}\label{asymmetric_authentication}

فرض کنید اثبات‌کننده و تایییدکننده یک کلیدعمومی
$pk$
را به‌اشتراک گذاشته و تنها اثبات‌کننده از کلیدخصوصی
$sk$
 وابسته به کلیدعمومی آگاهی دارد. همچنین فرض کنید 
$S_{sk}$	
 بیانگر الگوریتم امضا با کلیدخصوصی
$sk$
و
$V_{pk}$
الگوریتم تاییدساز وابسته به الگوریتم امضا می‌باشد که از کلیدعمومی
$pk$
استفاده می‌کند.
مطابق با شکل د، در آغاز پروتکل شناسایی ، تاییدکننده چالش 
$c~ {\in}_R ~ \{0,1\}^k$
را برای اثبات‌کننده ارسال می‌کند و درادامه اثبات‌کننده در جواب چالش دریافت‌شده، مقدار
$r = S_{sk}(c)$
را برمی‌گرداند. درادامه تاییدکننده بررسی می‌کند که آیا خروجی
$V_{pk}(c,r)$
معتبر می‌باشد یا خیر. اگر خروجی معتبر باشد بیانگر آن است که مقدار
$r$
درواقع امضای پیام
$c$
تحت کلبدعمومی
$pk$
می‌باشد.
\\
\\
برای مقاومت دربرابر حمله‌ی استراق‌سمع لازم است که طرح امضای دیجیتال در برابر حمله‌ی پیام‌  شناخته شده ایمن باشد و برای جلوگیری از حمله‌ی تاییدکننده‌متقلب لازم است تا این طرح در برابر حمله‌های پیام‌انتخابی انطباقی ایمن باشد.
\subsection{پروتکل شناسایی دانش صفر}\label{zk_id_protocol}\LTRfootnote{ZERO-KNOWLEDGE IDENTIFICATION PROTOCOLS}

طرح بخش قبل زمانی ایمن است که از طرح‌های احرازهویت یا رمزنگاری بسیارقوی(مثل طول کلید بزرگ و ..) استفاده شود.همچنین هزینه‌ی مقاومت در برابر حمله‌ی پیام انتخابی تطبیقی برای یک طرح امضای دیجیتال نیز بسیار بالا می‌باشد.‌علاوه‌براین محاسبه‌ی پاسخ چالش نیز برای اثبات‌کننده  ممکن است هزینه‌بر باشد. بنابراین برای رفع مشکلات طرح‌های قبلی به معرفی یک پروتکل جدید می‌پردازیم.
\\
در این بخش به معرفی پروتکل شناسایی دانش‌صفر می‌پردازیم که یکی از ویژگی‌های باارزش آن این است که با هر تلاش یک تاییدکننده‌متقلب، هیچ اطلاعات مفیدی از اثبات‌کننده(ی‌معتبر) استخراج نمی‌شود. به‌بیان‌دیگر واژه‌ی دانش‌صفر بیانگر این  حقیقت است که تاییدکننده‌‌متقلب با کسب هر اطلاعاتی که از طریق تعاملاتی که با اثبات‌کننده دارد، با اطلاعاتی که تاییدکننده‌متقلب از طریق خودش بدون تعامل با اثبات‌کننده تولیدشده است تفاوتی ندارد. 
به‌عبارت‌دیگر ممکن است پیامی که که‌بوسیله‌ی اثبات‌کننده ارسال شده است قابل شبیه‌سازی 
\LTRfootnote{simulated}
باشد که درعمل اثبات‌کننده در آن نقشی نداشته باشد.
یک تاییدکننده‌صدق به‌هرحال متقاعد می‌شود که اثبات‌کننده کلیدخصوصی را می‌داند.؟؟؟

\subsubsection{پروتکل دانش‌صفر اشنور}\label{schnore_zk_protocol}\LTRfootnote{SCHNORE ZERO-KNOWLEDGE PROTOCOL}
\\
 از نمونه‌های موجود پروتکل دانش‌صفر می‌توان به پروتکل اشنور اشاره کرد که براساس  لگاریتم گسسته  طراحی شده است.
\\
فرض کنیم
$\langle g \rangle $
یک گروه از درجه‌ی 
$n$
(که یک عدد اول بسیار بزرگ است) می‌باشد و همچنین 
$\boldsymbol{x} ~{\in}_R ~\mathbb{Z}_n$
 کلیدخصوصی و 
$\boldsymbol{h} = g^x$
نیز کلیدعمومی اثبات‌کننده می‌باشد.تاییدکننده درحین پروتکل ثبت‌‌نام، کلیدعمومی 
$h$
را دریافت می‌کند.یک دور از پروتکل اشنور در شکل ۲۰ نشان داده شده است. درمجموع این پروتکل، به تعداد
$k$
بار به‌صورت پی‌در‌پی بین اثبات‌کننده و تاییدکننده تکرار می‌شود.
$k$
یک پارامتر امنیتی می‌باشد.یک ساختار سه قسمتی به‌مانند اکثر پروتکل‌های دانش‌صفر در هربار تکرار پروتکل اشنور به‌صورت انتقال پیام انجام می‌شود؛ 
\begin{itemize}
\item {
 پیام اول،
$a$
 می‌باشد که دراصل تعهدی برای 
$u$
می‌باشد
}
\item {
 پیام دوم، 
$c$
است که چالش نامیده می‌شود
}
\item {
و پیام سوم،  
$r$
می‌باشد که پاسخ نامیده می‌شود
}
\end{itemize}~
\subsubsubsection{ویژگی صداقت}\label{soundness_property}
\\
درابتدا می‌خواهیم این بحث را شروع کنیم که چگونه پروتکل  اشنور، تاییدکننده را متقاعد می‌سازد که اثبات‌کننده واقعا کلیدخصوصی یعنی
$x = log_g^h$
را می‌داند.
 این خاصیت، ویژگی 
 \textbf{صداقت}
  نامیده می‌شود. و اگر تاییدکننده مجاب شود که اثبات‌کننده راست می‌گوید بنابراین ویژگی صداقت این پروتکل برآورده می‌شود.
\\
  اگر اثبات‌کننده مقدار
$x$
 را نداند، بهترین کاری که می‌تواند انجام دهد این است که اعلان
$a$
را چنان تولید کند که پاسخ
$r$
 برای حالت‌های 
$c=0$
و
$c=1$
درست عمل کند. بدین منظور یک اثبات‌کننده متقلب، برای مهیا کردن پاسخ چالش 
$c=0$
، مقدار اعلان را به‌صورت 
$a = g^u$
درنظر می‌گیرد و 
$r=u$
را ارسال می‌کند. همچنین برای پاسخ به چالش
$c=1$
 می‌تواند اعلان
$a = {g^u}/ h$
 را تنظیم و
$r=u$
 را ارسال کند؛ درادامه هنگام تاییدسازی رابطه‌ی
$g^r = ah$
 رخ خواهد داد.
\\ 
 نکته مهمی که باید در اینجا ذکر شود آن است که اثبات‌کننده بدون دانستن کلیدمخفی
$x$
  هیچ‌گاه نمی‌تواند پاسخی هم برای 
$c=0$
 و هم برای
$c=1$
مهیا کند. به‌عبارت‌دیگر فرض کنید بعد از ارسال اعلان
$a$
، اثبات‌کننده‌‌ای قادر باشد به هر دو چالش
$c=0$
و
$c=1$
به‌درستی پاسخ دهد. این مطلب به این معنی است که اثبات‌کننده قادر است تا دو پاسخ
$r_0$
و
$r_1$
را تولید کند که به‌ترتیب متناسب با چالش‌های
$c=0$
و
$c=1$
می‌باشد. با این حساب، مقادیر
$a$ 
و
$r_0$
و
$r_1$
رابطه‌ی
$$ g^{r_0} = a , \quad g^{r_1} = ah $$
را موجب می‌شوند که اشاره به رابطه‌ی زیر دارد:
$$ h = g^{r_1 - r_0} $$
و رابطه‌ی بالا دراصل نشان‌دهنده‌ی آن است که اثبات‌کننده عملا مقدار
$x$
را می‌داند. دلیل این امر هم آن است که رابطه‌ی زیر برقرار است:
$$ x \equiv {r_1 - r_0} \pmod n $$
درنتیجه با هربار تکرار پروتکل اشنور، اثبات‌کننده‌متقلب به‌احتمال
$ 50 \% $
موفق می‌شود تا به یکی از چالش‌ها به درستی جواب دهد ولی با تکرار 
$k$
بار، احتمال موفقیت اثبات‌کننده‌متقلب حداکثر به‌احتمال
$2^{-k}$
خواهد بود که با این احتمال بسیار کم عملا مشخص می‌شود که اثبات‌کننده‌ی متقلب هیچ شانسی برای پیروز شدن در این پروتکل را ندارد و درنتیجه ویژگی صداقت ضمانت حقیقی بودن اثبات‌کننده و اشراف به دانش موردنظر رادر این پروتکل تضمین می‌کند. 
\\
\subsubsubsection{ویژگی دانش‌صفر}\label{zero_knowledge_property}
\\
دراین قسمت می‌خواهیم نشان دهیم که پروتکل اشنور ویژگی دانش‌صفر را نیز برآورده می‌کند.
یک تاییدکننده‌ی متقلب می‌تواند چندین‌بار برای به‌دست‌آوردن گفگتگوهای
$(a;c;r)$
در هریک از مراحل پروتکل شناسایی مشغول شود. عبارت چندین‌بار به‌معنی حداکثر
$\mathcal{O}(k^\lambda)$
برای ثابت
$\lambda \in \mathbb{N}$
(محدود چندجمله‌ای در پارامترامنیتی
$k$)
می‌باشد.
علاوه‌براین تاییدکننده‌ی متقلب می‌تواند تعاملات
$(a;c;r)$
زیادی را به‌دست آورد.بااین حال تاییدکننده‌ی متقلب می‌تواند تعاملات بالا را بدون تعامل با  اثبات‌کننده به‌دست آورد. دراین حالت تعاملات به‌دست‌آمده‌ی
$(a;c;r)$
را نسخه‌ی شبیه‌سازی شده می‌نامیم که تاییدکننده‌متقلب،نقش تاییدکننده‌ی واقعی و اثبات‌کننده‌ی واقعی را خود تنها بازی می‌کند.
\\
در قدم اول ویژگی دانش‌صفر را برای یک تاییدکننده‌ی صادق
$\mathcal{V}$
نشان می‌دهیم که در این حالت تاییدکننده یک چالش
$c$
را به‌صورت اتفاقی از
$\{0,1\}$
انتخاب می‌کند.درادامه دو الگوریتم چندجمله‌ای احتمالاتی یکی برای تعاملات واقعی و دیگری برای تعاملات شبیه‌سازی شده ارائه می‌کنیم:
\\
sdsdsd
\\
هردو الگوریتم، تعاملات
$(a;c;r)$
را به‌صورت تصادفی و موردپذیرش تولید می‌کنند با این حال احتمال آن‌که تعامل شبیه‌سازی شده چندتایی
$(A;C;R) \in \langle g \rangle \times \{0,1\} \times \mathbb{Z}_n$
که در عبارت
$g^R = Ah^C$
صدق کند به‌صورت زیر می‌باشد:??
$$Pr[(a;c;r)=(A;C;R)] = \frac{1}{2n}$$
با این تفاوت که که تعاملات واقعی دسترسی به کلیدخصوصی 
$x$
را منجر می‌شود در صورتیکه تعاملات شبیه‌سازی شده منجر به تولید کلیدعمومی
$h$
 می‌شود.
\\
\remark
حقیقت این است که یک پروتکل شناسایی در برابر یک تاییدکننده صادق(حقیقی)، ویژگی دانش‌صفر را در خود دارد که درنتیجه‌ی آن هر حمله‌ به کلید کاهش می‌یابد؟؟؟. به‌خصوص آن‌که استراق‌سمع تعاملات میان اثبات‌کننده صادق با تاییدکننده هیچ اطلاعاتی در مورد کلیدخصوصی اثبات‌کننده افشا نمی‌کندُ حتی اگر کلیدعمومی را نیز دراختیار داشته باشیم؟؟؟.
\\
\\
در قدم بعدی ویژگی دانش‌صفر را برای هر حالت عمومی برای هر تاییدکننده‌ی متقلب 
$\mathcal{V}^*$
با زمان چندجمله‌ای احتمالاتی درنظر می‌گیریم. در اینجا از تاییدکننده‌ی ‌متقلب
$\mathcal{V}^*$
که یک نوع ماشین تورینگ احتمالاتی می‌باشد به‌عنوان یک جعبه‌سیاه
\LTRfootnote{black-box}
قابل چاپ مجدد
\LTRfootnote{rewindable}
استفاده خواهیم کرد، به‌این معنی که :
\begin{itemize}
\item 
دسترسی 
$\mathcal{V}^*$
فقط در درون جعبه سیاه می‌باشد، به این معنی که تبادل پیام‌ها با 
$\mathcal{V}^*$
از طریق ورودی خودش و خروجی نوار می‌باشد.
\item 
حالت
$\mathcal{V}^*$
را می‌توانیم به هر حالت قبلی بازگردانی کنیم.
\end{itemize}
از قسمت ۱٫۲٫۱ یادآوری می‌کنیم که تنظیمات یک ماشین تورینگ احتمالاتی از طریق حالت قسمت کنترل متناهی معین می‌شود. محتویات نوار به موقعیت هدر نوار بستگی دارد.از طریق بازگردانی
$\mathcal{V}^*$
می‌توانیم برای چندین ورودی‌ تست کنیم تا خروجی دلخواه ما به‌دست آید. 
\\
fsdfsdfsdf
\\
در مرحله‌ی شش‌ام شبیه‌سازی، از آنجا که
$c \in_R \{0,1\}$
، احتمال آن‌که 
$c=c'$
دقیقا برابر با
$1/2$
می‌باشد.بنابراین به‌طور متوسط بعداز دو مرحله‌ تکرار می‌توان تعاملات
$(a;c;r)$
را شبیه‌سازی کرد. نتیجه آن‌که، مهم نیست تاییدکننده‌متخاصم
$\mathcal{V}^*$
با چه الگوریتمی سعی در استخراج اطلاعات مهم از اثبات‌کننده می‌کند، از همین الگوریتم می‌توان به تولید تعامل توزیع‌شده منحصربه‌فرد بدون نیاز داشتن به همکاری با تاییدکننده نیز پرداخت.
البته قابل ذکر است این وقایع درحالی رخ می‌دهد که تعاملات واقعی از طریق کلیدخصوصی 
$x$
به‌عنوان ورودی تولید شده‌اند و تعاملات شبیه‌سازی شده از طریق کلیدعمومی 
$h$
به‌عنوان ورودی تولید شده‌اند.

\subsubsection{پروتکل اشنور}\label{schnorr_protocol}

پروتکل شکل ۴٫۲ یک پروتکل دانش‌صفر می‌باشد.بنابرآنچه قبلا بحث کردیم، احتمال موفقیت اثبات‌کننده‌ی متقلب 
$50\%$
می‌باشد.با استفاده از 
$k$
تکرار پی‌در‌پی این پروتکل، ویژگی دانش‌صفر قاعدتا حفظ خواهد شد اما احتمال موفقیت اثبات‌کننده‌ی متقلب به 
$2^{-k}$
کاهش پیدا خواهد کردکه این عدد مقدار بسیار ناچیز به عنوان یک تابع از مقدار امنیتی
$k$
خواهد بود.
\\

به‌دلیل آنکه هم اثبات‌کننده و هم تاییدکننده نیاز به  محاسبه‌ی 
$\mathcal{O}(k)$
عمل توان در گروه
$\langle g \rangle$
خواهند داشت بنابریان پیچیدگی محاسباتی پروتکل نتیجه نسبتا بالا می‌باشد. بنابریان اشنور همچنین استفاده ؟؟ را در شکل ۴٫۳ ارائه کرده است(که به عنوان پروتکل اشنور سناخته می‌شود). در این پروتکل، تاییدکننده چالش خود را از یک  میدان بسیار بزرگ به عنوان مثال
$c \in \mathbb{Z}_n$
انتخاب می‌کند.
\\


ویژگی صداقت پروتکل اشنور می‌تواند همچون بالا آنالیز شود.فرض کنید یک اثبات‌کننده  پس از ارسال 
$a$
به‌عنوان اعلان،
قادر باشد تا به حداقل دو چالش 
$c$
و
$c'$ 
با این فرض که
$c \neq c'$
،به‌درستی جواب دهد. درنتیجه اثبات‌کننده قادر خواهد بود تا دو مبادله‌ی
$(a;c;r)$
و
$(a;c';r')$
را تولید کند. البته ذکر این نکته لازم است که این اوامر در صورتی رخ می‌دهد که اثبات‌کننده واقعا لگاریتم گسسته زیر را بداند:
$$ x = \log_g h $$
همچنین 
$$ g^r = ah^c , \quad g^{r'} = ah^{c'}  $$
نشان‌دهنده‌ی آن است که:
$$ h = g^{(r-r') / (c-c')} $$
بنابراین 
بعد از ارسال اعلان 
$a$
اگر اثبات‌کننده کلیدخصوصی
$x$
را نداند آنگاه حداکثر می‌تواند به یک چالش به‌درستی پاسخ درست دهد.از آنجا که 
$n$
حالت ممکن برای چالش وجود دارد(چالش‌ها از یک میدان بسیار بزرگ انتخاب می‌شوند)، احتمال موفقیت 
$1 / n$
می‌باشد که عدد بسیار کوچکی می‌باشد.
\\
~
figure 4.3
~
\\
ویژگی دانش‌صفر نیز همچنین مي تواند در پروتکل اشنور همچون بالا برای یک تاییدکننده صادق 
$\mathcal{V}$
اثبات شود. مراودات دو گفگتگوی واقعی(تولیدشده توسط کلیدخصوصی 
$x \in \mathbb{Z}_n$)
و گفتگوی شبیه‌سازی شده‌(تولیدشده توسط کلیدعمومی
$h \in \langle g \rangle$)
به‌صورت زیر می‌باشند:
$$\{ (a;c;r) :‌u,c \in_R \mathbb{Z}_n ;~ a \leftarrow g^u ;~ r {\leftarrow}_n u+cx \}$$
$$\{ (a;c;r) : c,r \in_R \mathbb{Z}_n ;~ a \leftarrow g^rh^{-c} \}$$
این مراودات یکتا هستند،


\newpage
\subsection{اثبات دانش‌صفر}\label{zero_knowledge_proof}
اثبات دانش‌صفر یک کلاس عمومی از پروتکل‌هایی است که بین دو بخش به‌نام های اثبات‌کننده و تاییدکننده صورت می‌پذیرد.از طریق اثبات دانش‌صفر، اثبات‌کننده تاییدکننده را متقاعد می‌کند که یک عبارت داده شده معتبر است بدون آن‌که هیچ اطلاعات ارزشمندی از حقیقت عبارت را افشا کند.
\\
\\
در پروتکل‌های شناسایی دانش‌صفر، عبارت اثبات‌شده شبیه چیزی مثل "من کلیدخصوصی این کلیدعمومی را می‌دانم" می‌باشد. البته عبارت می‌تواند بزرگتر و بیشتر هم باشد،به‌عنوان مثال "من کلیدخصوصی را برای این کلید عمومی یا آن کلیدعمومی می‌دانم،که در هر حالت کلیدهای خصوصی می‌توانند متفاوت باشند."
\\
\\
درحقیقت، نظریه‌ی اثبات دانش‌صفر گویای این است که هر جمله‌ی چندجمله‌ای غیرمعین
\LTRfootnote{NP-statment}
می‌تواند به طور موثر در دانش‌صفر اثبات شود.

\subsubsection{پروتکل زیگما}\label{sigma_protocols}

مفهوم یک پروتکل زیگما، پروتکل شناسایی (ارائه شده توسط اشنور) را تعمیم می‌دهد. یک پروتکل زیگما فقط باید دربرابر تاییدکننده‌ی صادق، ویژگی دانش‌صفر را دارا باشد.؟؟
\\
به‌دلایل فنی، یک پروتکل زیگما در واقع نیاز دارد تا یک دانش‌صفرِ ‌تاییدکننده‌ی صادقِ ویژه باشد، به‌این‌معنی که شبیه‌ساز بتواند چالش
$c$
را به‌عنوان یک ورودی اضافی انتخاب و یک گفگتگو براساس چالش مشخص
$c$
تولید کند.تمام پروتکل‌های ذکرشده در بالا همگی دانش‌صفر تاییدکننده‌صادق ویژه می‌باشند(بررسی آن آسان است.)
\\

$R = \{ (v;w) \subseteq V \times W \}$
را یک رابطه‌ی باینری فرض کنیم که 
$v \in V$
معرف ورودی عمومی اثبات‌کننده و تاییدکننده می‌باشد و 
$w \in W$
معرف
\textbf{شاهد}
است که ورودی خصوصی اثبات‌کننده می‌باشد.
  
\definition 
یک پروتکل زیگما برای رابطه‌ی
$R$
، پروتکلی طبق شکل ۵٫۱ است بین یک اثبات‌کننده
$\mathcal{P}$
و یک تاییدکننده
$\mathcal{V}$
که دارای ویژگی‌های زیر می‌باشد:
\begin{itemize}
\item{ \textbf{کامل بودن.}}
اگر اثبات‌کننده
$\mathcal{P}$
و تاییدکننده
$\mathcal{V}$
از پروتکل پیروی کنند آنگاه تاییدکننده همواره خروجی 
\textbf{موافقت}
را تولید می‌کند.
\item{\textbf{صداقت ویژه.}}
یک الگوریتم چندجمله‌ای احتمالاتی 
$E$
(استخراج‌کننده) وجود دارد که با دریافت هر 
$v \in V$
و هر جفت گفتگوهای موردپذیرش
$(a;c;r)$
و
$(a';c';r')$
که
$c \neq c'$
همواره شاهد
$w$
را محاسبه می‌کند که رابطه‌ی
$(v;w) \in R$
 برقرار می‌باشد.
\item{\textbf{دانش‌صفرِ تاییدکننده‌صادقِ ویژه.}}
یک الگوریتم چندجمله‌ای احتمالاتی 
$S$
(شبیه‌ساز) وجود دارد که با دریافت هر
$v \in L_R$
و چالش
$c \in C$
، یک گفتگوی
$(a;c;r)$
را با همان توزیع احتمال تولید گفتگو بین اثبات‌کننده‌ی صادق
$\mathcal{P}$
و تاییدکننده‌ی صادق
$\mathcal{V}$
با ورودوی عمومی 
$v$
و چالش
$c$
تولید می کند. البته قابل ذکر است که اثبات‌کننده‌ی
$\mathcal{P}$
از شاهد
$w$
برای برقراری رابطه‌ی
$(v;w) \in R$
استفاده می‌کند.
\end{itemize}
همچنین ذکر این نکته نیز لازم است که اگر 
$C$
شامل یک عضو باشد آنگاه پروتکل زیگما را بدیهی می‌نامیم.
\\

ویژگی صداقت ویژه اشاره یه این نکته دارد که احتمال موفقیت اثبات‌کننده‌ی متقلب حداکثر
$1/n$
است که در اینجا
$n$
نشان‌دهنده‌ی اندازه‌ی فضای چالش‌ها یا
$C$
می‌باشد.ازاین‌رو بااین فرض که 
$n$
نسبتا بزرگ می‌باشد آنگاه یک پروتکل زیگما دانش یک شاهد
$w$
برای یک ورودی عمومی
$v$
را اثبات می‌کند.

\proposition
پروتکل شکل ۴٫۳ یک پروتکل زیگما برای رابطه‌ی 
$\{ (h;x) : h=g^x \}$
می‌باشد.
\refproof~
\\
\begin{itemize}
\item{کامل بودن.}
این ویژگی به‌راحتی قابل بیان است:
$$g^r = g^{u+cx} = g^u(g^x)^c = ah^c$$
قابل ذکر است که اثبات‌کننده 
$r = u+cx$
‌را محاسبه می‌کند.
\item{صداقت ویژه.}
با دریافت دو گفتگوی موردپذیرش 
$(a;c;r)$
و
$(a';c';r')$
درحالیکه
$c \neq c'$
، آنگاه داریم:
\begin{align}
 &\quad ~~ g^r = ah^c, g^{r'} = ah^{c'} \\
 & \Rightarrow g^{r-r'} = h^{c-c'}\\
 & \Longleftrightarrow h = \frac{r-r'}{c-c'}
\end{align}
ازاین‌روز شاهد به صورت 
$ x = (r-r') / (c-c') $
 به‌دست می‌آید.
\item{دانش‌صفرِ تاییدکننده‌ی صادقِ ویژه.}
برای نشان دادن این ویژگی نیز، با دریافت دو توزیع از گفتگوها، یکی با تاییدکننده‌ی صادق(که با اجرای پروتکل رخ می‌دهد و شاهد 
$x$
به‌عنوان ورودی استفاده می‌شود) و دیگری گفتگوی شبیه‌سازی شده(که با انحراف پروتکل، از کلیدعمومی
$h$
به‌عنوان ورودی استفاده می‌شود) داریم:
\begin{align}
& \{ (a;c;r) : u \in_R \mathbb{Z}_n ; a \leftarrow g^u ; r {\leftarrow}_n u+cx \},\\
& \{ (a;c;r) : r \in_R \mathbb{Z}_n ; a \leftarrow g^rh^{-c} \}
\end{align} 
که با دریافت چالش
$c$
دلخواه، این توزیع‌ها یکتا هستند(هر توزیع دقیقا با احتمال
$1/n$
رخ می‌دهند). 
\end{itemize}~
\\
\\
\\
\subsubsection{پروتکل زیگماي غیرتعاملی}\label{non_interactive_sigma_protocol}

یادآوری می‌کنیم که اساسا دو نمونه از طرح‌های احرازهویت وجود دارند:طرح‌های احرازهویت تعاملی(مانند طرح‌های شناسایی) و طرح‌های احرازهویت غیرتعاملی(مانند امضای دیجیتال). 
به‌طور مشابه در اثبات دانش‌صفر نیز دو فرم وجود دارد:طرح‌های اثبات دانش‌صفر تعاملی و طرح‌های اثبات دانش‌صفر غیرتعاملی.
یک طرح اثبات غیرتعاملی شامل یک پروتکل است که بوسیله‌ی آن یک اثبات‌کننده، یک تاییدکننده را متقاعد می‌سازد که یک بیانه خاص دارد. در مقابل یک طرح اثبات غیرتعاملی شامل دو الگوریتم است  یکی برای آن‌که اثبات‌کننده یک اثبات را برای یک بیانه خاص تولید کند و  الگوریتم دیگر برای آن است که تاییدکننده صحت اثبات داده شده را بررسی کند. 
\\

یک روش ساده و موثر وجود دارد که هر پروتکل زیگما را غیرتعاملی می‌سازد، این روش توسط فیات و شمیر ارائه شده‌اند و به طرح فیات-شمیر شناخته می‌شود.
\\
????
\\
یک ویژگی متمایز اثبات زیگمای غیرتعاملی این است که هر موجودیتی می‌تواند نقش اثبات‌کننده را ایفا کند. به‌عنوان نتیجه می‌توان گفت که اثبات زیگمای غیرتعاملی می‌تواند توسط هر موجودیتی مستقلا تایید شود( مانند امضای دیجیتال که توسط هرشخصی که علاقه به تاییدسازی آن دارد انجام می‌شود).
\\
\\
\subsubsection{امضای دیجیتال ازطریق پروتکل زیگما}\label{ds_from_sigma_protocol}

یک طرح امضای دیجیتال شامل سه الگوریتم می‌باشد:
\begin{itemize}
	\item 
	الگوریتم تولید کلید
	\item 
	الگوریتم تولید امضا
	\item 
	الگوریتم تاییدسازی امضا
\end{itemize}
با یک مثال می‌خواهیم نشان دهیم تا چگونه از هر پروتکل‌ زیگما(که برای شناسایی استفاده می‌شود) یک طرح امضای دیجیتال معادل بسازیم. برای این منظور لازم است تا از تابع هش
$H$
استفاده کنیم.
\\

در پروتکل اشنور(شکل ۴٫۳)، برای اثبات دانش
~$\pmb{x = \log_g h}$~
از کلیدخصوصی 
$x$
و کلیدعمومی
$h$
استفاده می‌کنیم.
طرح امضای دیجیتال اشنور از طریق اعمال پروتکل فیات-شمیر به پروتکل اشنور به‌دست می‌آید. این به‌معنی ان است که چالش
$c$
ازطریق مقدار هش به‌دست‌ آمده از طریق تابع هش با دو ورودی پیام
$M$
و اعلان
$a$
به‌دست می‌آید،
$c \leftarrow H(a,M)$.
البته قابل ذکر است که چالش
$c$
به‌صورت کاملا تصادفی و مستقل از اعلان
$a$
به‌دست می‌آید. در این روش هیچ تعاملی بین تاییدکننده و تاییدکننده نیاز نیست  و تاییدکننده خود به تولید چالش
$c$
می‌پردازد. امنیت این طرح امضا به تابع هش
$H$‌‌
به‌عنوان یک اوراکل تصادفی می‌باشد.از آنجا که 
$H$
یک تابع تصادفی می‌باشد، بنابریان به‌نظر می‌رسد  مقدار چالش
$c= H(a,M)$
با هر پیام 
$M$
متفاوتی برای امضا، به همان میزان تصادفی باشد که اثبات‌کننده چالش را از تاییدکننده‌ی صادق با تعامل دریافت می‌کند. علاوه بر این لازم است تا اثبات‌کننده قبل از محاسبه‌ی چالش
$c$
، اعلان 
$a$
را انتخاب کند.
همچنین فرض می‌شود که 
$2^k \leq n$.
درنتیجه رشته بیتی در
$\{0,1\}^k$
می‌تواند با مقدارهای عددی در 
$\{0,1, \dots , 2^k-1\} \subseteq \mathbb{Z}_n$
یکتا باشد؟؟؟.
طرح امضای اشنور از سه الگوریتم زیر تشکیل شده است، که تمام کاربران گروه
$\langle g \rangle$
با مرتبه‌ی عدد اول
$n$ 
و تابع هش
$H : \{0,1\}^* \rightarrow \{0,1\}^k$
به اشتراک می‌گذارند.
\begin{itemize}
	\item{\textbf{الگوریتم تولید کلید.}}
	یک جفت کلید
	$(h;x)$
	با انتخاب کلیدخصوصی
	$x \in_R \mathbb{Z}_n$
	و مقدار کلیدعمومی
	$h$
	که از طریق 
	$h \leftarrow g^x$
	به‌دست می‌آید، ساخته می‌شوند.
	\item{\textbf{الگوریتم تولید امضا.}}
	با ورودی پیام
	$M$
	و کلیدخصوصی
	$x$
	و انتخاب
	$u \in_R \mathbb{Z}_n$
	و محاسبه‌ی 
	$a \leftarrow g^u$
	،
	$c \leftarrow H(a,M) $
	،
	$ r \leftarrow_n u + cx $
	، امضای پیام
	$M$
	، زوج
	$(c,r)$
	می‌باشد.
	\item{\textbf{الگوریتم تاییدساز امضا.}}
	با دریافت پیام
	$M$
	و زوج
	$(c,r)$
	به‌عنوان امضا و یک کلیدعمومی
	$h$
	،امضای
	$(c,r)$
	بر پیام
	$M$
	مورد پذیرش است اگر تساوی زیر برقرار باشد:
	$$ c = H(g^rh^{-c}, M) $$
\end{itemize}~
\\

کاملا واضح است که طرح امضای اشنور بسیار کارآمد است.هزینه‌ی محاسباتی تولید امضا وابسته به زمان انجام یک عمل توان برای تشکیل
$g^u$
می‌باشد ؟؟ .
هزینه محاسباتی تاییدسازی امضا وابسته به زمان انجام دوعمل توان رسانی
$g^rh^{-c}$
می‌باشد







