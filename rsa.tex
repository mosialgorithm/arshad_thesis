برای روشن تر شدن بحث یک مثال از امضای دیجیتال بر پایه‌ی مسئله 
$RSA$
ذکر می‌کنیم.
\\
\\
\begin{example}\label{ex_rsa_sig}
	
\textbf{
 امضای دیجیتال 
$\textbf{RSA}$
}\label{ex_ds} {

}

در ابتدا پوریا(به عنوان امضاکننده) دو عدد اول بزرگ  
$p$
و
$q$
را به صورت مخفی انتخاب و عدد
$N = p\cdot q$
را به همراه
$v$
 به عنوان کلید‌عمومی(تاییدساز)  منتشر می‌کند. پوریا با دانش تجزیه عدد
$N$
، کلیدخصوصی خود یعنی 
$s$
را به دست می‌آورد

$$ sv \equiv 1 \pmod {(p-1) (q-1)}$$

درادامه پوریا برای امضای یک سند از کلید خصوصی خود یعنی
$s$
استفاده می‌کند و برای تایید امضا باید از کلید عمومی(تاییدساز)
$v$
استفاده شود.
\RTLfootnote{
اگر سیستم رمزنگاری 
$RSA$
را درنظر بگیریم آنگاه 
$v$
برای رمزنگاری پیام و 
$s$
برای رمزگشایی پیام استفاده می‌شود.
}
برای امضای سند دیجیتالی 
$D$
، فرض می‌کنیم که در محدوده‌ی 
$1 < D < N$
می‌باشد و در ادامه پوریا مقدار زیر که معرف امضا می‌باشد را محاسبه و منتشر می‌کند
$$S \equiv D^s \pmod N$$ 
ویکتور(تاییدکننده) برای تایید اعتبار امضای 
$S$
روی سند
$D$
محاسبه زیر را انجام می‌دهد
$$ S^v \pmod N $$
و بررسی می‌کند که آیا جواب برابر با با 
$D$
می‌باشد یا خیر . دلیل این بررسی وجود فرمول اویلر می‌باشد که به رابطه زیر ختم می‌شود
$$ S^v \equiv D^{sv} \equiv D \pmod{N} $$

\end{example}