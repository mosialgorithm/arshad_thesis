\chapter{خم بیضوی}
فرض کنید
$k$
یک میدان بوده و 
$f(x,y)$
یک چندجمله‌ای با ضرایب روی میدان
$k$
باشد، هم‌چنین فرض کنید:
$$C = \{ (x,y) \in k^2 \mid f(x,y) = 0 \}$$
نقطه‌ی 
$p=(x,y) \in C$
را نقطه‌ی تکین(منفرد) خم 
$C$
گوییم هرگاه:
$\frac{\sigma f}{\sigma x} (p) = \frac{\sigma f}{\sigma y} (p) = 0 $
درغیراینصورت 
$p$
را نامنفرد گوییم.
\\
اگر خم
$C$
هیچ نقطه‌ی تکینی نداشته باشد، همورا نامیده می‌شود.
\example
نقاط تکین خم با معادله‌ی
$f(x,y) = y^2 - x^3 + 3x$
را در صورت وجود بیابید.
\\
\begin{equation*}
\begin{cases}
&\frac{\sigma f}{\sigma x} = -3x+3=0 \longrightarrow x = \pm 1 \\
&\frac{\sigma f}{\sigma y} = 2y=0 \longrightarrow y=0 
\end{cases}
\end{equation*}
پس 
$P(1,0)$
و
$Q(-1,0)$
کاندیدای نقطه‌ منفرد خم هستند. اما به‌راحتی می‌توان دید که هیچ کدام از نقاط
$P$
و
$Q$
روی خم قرار ندارند پس این خم نقطه تکین ندارد و لذا هموار است.
\\
\definition
فرض کنید
$K$
یک میدان باشد، معادله‌ی
$$ y^2 + a_1xy + a_3 = x^3 + a_2x^2 + a_4x+a_6 $$
که درآن
$a_1,a_2,a_3,a_4,a_6 \in K$
یک معادله‌ی وایرشتراس نامیده می‌شود.
\\
\definition
هر خم هموار با معادله‌ی وایرشتراس بالا یک خم بیضوی نامیده می‌شود.
\\
\definition
معادله‌ی
$y^2 = x^3+Ax+B$
معادله‌ی وایرشتراس کوتاه نامیده می‌شود.
\\
نمودار خم با معادله‌ی وایرشتراس
$y^2 = x^3+1$
در میدان
$R$
به صورت زیر است:
% fiqure for elliptiC Curve
\\
\definition
مبین یک خم وایرشتراس کوتاه شده به‌صورت زیر می‌باشد:
$$\Delta = -(4A^3+27B^2)$$
\remark
خم
$y^2 = x^3+Ax+B$
هموار است اگر و تنها اگر 
$\Delta \ne 0$.
بنابراین یک خم بیضوی را می‌توان خمی با معادله‌ی وایرشتراس بالا و مبین غیرصفر تعریف کرد.
\section{
$j$-پایای یک خم
}
در این بخش فرض می‌کنیم 
$k$
میدانی با مشخصه‌ی مخالف ۲ و ۳ است. همچنین فرض می‌کنیم خم
$E$
دارای معادله‌ی وایرشتراس به فرم زیر باشد:
$$E:y^2=x^3+Ax+B$$
مبین این خم عبارت است از:
$$ \Delta = -(4A^3+27B^2) $$
اکنون $j$-پایای خم
$E$
را به‌صورت زیر تعریف می‌کنیم:
$$ j(E) = -1728\frac{4A^3}{\Delta} $$
\\
\example
فرض کنید
$$ E_1 : y^2 = x^3+x+1 $$
$$ E_2: y^2=x^3+4x+8 $$
در این صورت:
\\
در خم
$E_1$:
$$A=1,B=1 \Longrightarrow \Delta(E_1) = -(4A^3+27B^2) = -31 $$
$$ j(E_1)=-1728\frac{4A^3}{\Delta} = -1728\frac{4}{-31} = \frac{6912}{31} $$
در خم
$E_2$:
$$A=4,B=8 \Longrightarrow \Delta(E_2) = -(4A^3+27B^2) = -1984 $$
$$ j(E_2)=-1728\frac{4A^3}{\Delta} = -1728\frac{256}{-1984} = \frac{442368}{1984} $$
همان‌گونه که ملاحظه می‌شود، خم‌های 
$E_1$
و
$E_2$
دارای
$j$-پایای برابرند.
\section{یکریختی خم‌های بیضوی}
دو خم 
$E:y^2=x^3+Ax+B$
و
$E': y^2=x^3+A'x+B'$
را یکریخت گوییم هرگاه
$\mu \in {\bar{K}}^*$
موجود باشد که
$$ A'={\mu}^4A ~,~ B'={\mu}^6B $$
\example
فرض کنید
$E$
و
$E'$
دو خم روی 
$Q$
با معادلات زیر باشند:
$$ E:y^2=x^3+3x+5 ~,~  E':y^2=x^3+12x+40$$
دراینصورت
$A=3$
،
$B=5$
،
$A'=12$
،
$B'=40$
، با قرار دادن
$\mu = \sqrt{2}$
داریم:
$$ A'={\mu}^4 ~,~ B'={\mu}6B$$
پس خم‌های 
$E$
و
$E'$
روی
$Q$
 یکریخت نیستند درحالیکه روی
$\bar{Q}$
(یا روی 
$R$
یا روی
$Q\sqrt{2}$)
یکریخت‌اند.
\theorem
فرض کنید
$K$
یک میدان با مشخصه‌ی مخالف ۲ و ۳ باشد و 
$$ E:y^2=x^3+Ax+B ~,~ E': y^2=x^3+A'x+B' $$
دو خم روی
$K$
باشند، در اینصورت:
\begin{center}
$E$
و
$E'$
یکریخت‌اند اگر وتنها اگر 
$j(E)=J'(E')$
\end{center}
\refproof
اثبات را بیاورم؟؟؟؟
\section{درون‌ریختی خم‌های بیضوی}

اگر
$E$
یک خم بیضوی روی میدان
$K$
باشد انگاه
$\varphi : E(\bar{K}) \rightarrow E(\bar{K})$
یک درون‌ریختی است هرگاه:
\begin{enumerate}
\item{
$\varphi$
توسط تابع گویا بیان شده باشد، یعنی:
$\varphi(x,y) = (\varphi_1(x,y) , \varphi_2(x,y) )$
که درآن
$\varphi_1$
و
$\varphi_2$
توابع گویا هستند.
}
\item{
برای هر
$P$
و
$Q$
:
$\varphi(P+Q) = \varphi(P)+\varphi(Q)$
}
\item{
$\varphi(\infty) = \infty$
}
\end{enumerate}

\section{خم‌های بیضوی روی میدان‌های متناهی}

فرض کنید
$E$
یک خم بیضوی
$y^2=x^3+Ax+B$
روی میدان
$\mathbb{F}_q$
باشد. در این‌صورت 
$\mathbb{F}_q$-نقاط
روی خم عبارتنداز:
$$
 E(\mathbb{F}_q) = \{ (x,y) \in \mathbb{F}_q \times \mathbb{F}_q | y^2=x^3+Ax+B \} 
\Cup \{ \infty \}
$$
واضح است که:
$$ E(\mathbb{F}_q) \subseteq (\mathbb{F}_q \times \mathbb{F}_q) \Cup \{ \infty \} $$
و چون مجموعه‌ی سمت راست متناهی (از مرتبه‌ی
$q^2+1$)
است لذا مجموعه‌ی چپ یعنی
$E(\mathbb{F}_q)$
نیز متناهی است. پس:
$\# E(\mathbb{F}_q) \leq q^2+1$
بنابراین خم‌های بیضوی روی میدان‌های متناهی، متناهی اند.
\\
قضیه هسه کرانی برای تعداد عناصر 
$E(\mathbb{F}_q)$
معرفی می‌کند.
\theorem
اگر
$E$
یک خم بیضوی روی میدان
$\mathbb{F}_q$
باشد آنگاه:
$$ | q+1- \#E(\mathbb{F}) | \leq 2\sqrt{q} $$
به‌عبارت دیگر:
$$ -2\sqrt{q} \leq \# E(\mathbb{F}_q)  -(q+1) \leq +2\sqrt{q}$$
$$ (q+1)-2\sqrt{q} \leq \# E(\mathbb{F}_q) \leq (q+1)+2\sqrt{q}$$
$$ (\sqrt{q}-1)^2 \leq \# E(\mathbb{F}_q) \leq (\sqrt{q}+1)^2$$
\example
خم بیضوی با معادله‌ی وایرشتراس
$y^2=x^3+Ax+B$
روی میدان
$\mathbb{F}_{49}$
را درنظر بگیرید. در این‌صورت:
$$ 49+1-2\sqrt{49} \leq \# E(\mathbb{F}_{49}) \leq 49+1+2\sqrt{49}$$
$$ 36 \leq \# E(\mathbb{F}_{49}) \leq 64$$

\definition\label{ss}
$a_q=q+1- \# E(\mathbb{F}_q)$
، اثر خم نامیده می‌شود.
\\
\\
بنا به قضیه‌ هسه،
$|a_q| \leq 2\sqrt{q}$
یعنی
$ -2\sqrt{q} \leq a_q \leq 2\sqrt{q}$
، بنابراین در مثال قبل:
\\
$-14 \leq a_{49} \leq 49$.
\section{نقاط تابی در خم‌های بیضوی}

فرض کنید
$E$
یک خم بیضوی روی میدان
$\mathbb{F}_q$
باشد. برای عدد صحیح 
$n$
، درون‌ریختی زیر را در نظر می‌گیریم:
$$ [n]: E(\bar{\mathbb{F}_q}) \longrightarrow E(\bar{\mathbb{F}_q}) $$
هسته‌ی این درون‌ریختی عبارت است از:
$ker([n]) = \{ p \in E(\bar{\mathbb{F}_q}) ~|~ [n]p=\infty \}$
این مجموعه با 
$E[n]$
نمایش داده می‌شود، به‌عبارت دیگر:
$$ E[n] = \{ p \in E(\bar{\mathbb{F}_q}) ~|~ np=\infty \} $$
به‌راحتی می‌توان دید
$E[n]$
زیرگروهی از 
$E(\bar{\mathbb{F}_q})$
است.
\theorem 
اگر
$E$
یک خم بیضوی رو میدان
$\mathbb{F}_q$(
$q$
توانی از یک عدد اول) باشد آنگاه برای هر
$n$، 
$$ \# E({\mathbb{F}}_{q^n}) = q+1-({\alpha}^n +{\beta}^n) $$
که در آن
$\alpha$
و
$\beta$
ریشه‌های چندجمله‌ای زیر هستند:
$$x^2-a_qx+q=0$$

\example
فرض کنید خم
$E$
روی میدان
$\mathbb{F}_4$
توصیف شده باشد و 
$\# E(\mathbb{F}_4) = 6$.
حال می‌خواهیم
$\# E(\mathbb{F}_{16})$
را به‌دست آوریم، بنابراین:
$$q=4, a_q=q+1- \# E(\mathbb{F}_q) \Longrightarrow a_4=4+1-6=-1$$
در ادامه ریشه‌های چندجمله‌ای 
$x^2-a_qx+q=0$
را به‌دست می‌آوریم:
$$ x^2-(-1)x+4=0 \longrightarrow x^2+x+4=0, ~~ \Delta=b^2-4aC=1-16=-15 $$
\begin{equation*}
\text{ریشه‌ها}= \frac{-b \pm \sqrt{\Delta}}{2a} = \frac{-1 \pm \sqrt{-15}}{2} \Longrightarrow
\begin{cases}
& \alpha = \frac{-1- \sqrt{-15}}{2} \\
& \beta = \frac{-1+ \sqrt{-15}}{2} 
\end{cases}
\end{equation*}
بنابراین
$$ \# E(\mathbb{F}_{16}) = 16+1 -({\alpha}^2 + {\beta}^2) = 16+1-(-7)=24 $$

\theorem 
فرض کنید 
$E$
یک خم بیضوی روی میدان 
$k$
باشد، دراینصورت:
\begin{enumerate}
\item{
اگر
$char(k)=0$
یا
$char(k)=p$
که در آن
$p \nmid n$،
آنگاه:
$$E[n] \cong \mathbb{Z}_n \times \mathbb{Z}_n$$
}

\item{
اگر
$char(k)=p$
و
$p \mid n$،
آنگاه:
$$E[n] \cong \mathbb{Z}_m \times \mathbb{Z}_m$$
یا
$$E[n] \cong \mathbb{Z}_n \times \mathbb{Z}_m$$
که در آن
$n=p^e.m$
و
$(p,m)=1$
}
\end{enumerate}
\example
فرض کنید
$E$
یک خم بیضوی روی میدان
$\mathbb{F}_9$
باشد. ساختار
$E[5]$،
$E[6]$
و
$E[675]$
را به‌صورت زیر به‌دست می‌آوریم:
\\
محاسبه‌ی
$E[5]$:
\\
$char(\mathbb{F}_9)=3$
 بنابراین چون
 $3 \nmid 5$
 لذا:
 $$ E[5] \cong \mathbb{Z}_5 \times \mathbb{Z}_5 $$
 محاسبه‌ی
 $E[6]$:
 \\
$$
p=3 \mid 6=n \longrightarrow n=3 \times 2 \Rightarrow
E[6] \cong \mathbb{Z}_2 \times \mathbb{Z}_2 ~\text{یا}~ \mathbb{Z}_6 \times \mathbb{Z}_2
$$ 

محاسبه‌ی
$E[675]$:
\\
$$
p=3 \mid 675=n \longrightarrow n=3^3 \times 25 \Rightarrow
E[675] \cong \mathbb{Z}_{25} \times \mathbb{Z}_{25} ~\text{یا}~ \mathbb{Z}_{675} \times \mathbb{Z}_{25}
$$ 
\example
فرض کنید
$k$
میدانی با مشخصه‌ی
$p$
باشد. ساختار
$E[p]$
به‌صورت زیر خواهد بود:
\\
$$ n=p ~,~ p \mid n \Longrightarrow n=p^1 \times 1 \Longrightarrow m=1 $$
\begin{equation*}
\begin{cases}
& E[p] \cong  \mathbb{Z}_1 \times \mathbb{Z}_1 = \{(0,0)\} \Longrightarrow E[p]=\{ \infty \}\\
& E[p] \cong  \mathbb{Z}_p \times \mathbb{Z}_p
\end{cases}
\end{equation*}
\\
\definition
اگر
$E$
یک خم بیضوی روی میدان
$k$
(با مشخصه‌ی
$p$)
باشد و 
$E[p]={\infty}$،
آنگاه خم را سوپرسینگولار گوییم. در غیراینصورت ($E[p] \cong \mathbb{Z}_p$)
خم را معمولی گوییم.
\theorem 
فرض کنید 
$E$
یک خم بیضوی روی میدان
$ \mathbb{F}_q$
($q$
توانی از عدد اول
$p$)
باشد، دراینصورت:
$$ a_q \equiv 0 ~ (\mod p) ~~~\text{اگر و تنها اگر}~~~ \text{سوپرسینگولار است} E	 $$
\\
\remark
$$a_q = q+1- \# E(\mathbb{F}_q)$$
\begin{align*}
a_q \equiv 0 (\mod p) 
&	\iff q+1- \#E(\mathbb{F}_q) \equiv 0 (\mod p) \\
&	\iff 1- \#E(\mathbb{F}_q) \equiv 0 (\mod p)	\\
&	\iff \#E(\mathbb{F}_q) \equiv 1 (\mod p)
\end{align*}
\\
\section{زوجیت وایل}
فرض کنید 
$k$
یک میدان و 
$n$
عددی صحیح باشد بطوریکه 
$char(k)=0$
یا
$char(k)=p\nmid n$
همچنین فرض کنید
$E:y^2=x^3+Ax+B$
خم بیضوی روی میدان
$k$
باشد. در اینصورت نگاشت
$$ e_n : E[n] \times E[n] \longrightarrow \mu_n $$
با خواص زیر وجود دارد:
\begin{enumerate}
\item{
$e_n$
دوخطی است. بدین معنی که:
%==========================
\begin{enumerate}
\item{
$\forall S_1,S_2,T \in E[n] ~:~ e_n(S_1+S_2,T)=e_n(S_1,T)en(S_2,T)  $
}

\item{
$\forall S,T_1,T_2 \in E[n] ~:~ e_n(S,T_1+T_2)=e_n(S,T_1)en(S,T_2)  $
}
\end{enumerate}
%===========================

\lemma
اگر
$S \in E[n]$
آنگاه
$e_n(S,\infty)=1$
}
\item{
$e_n$
ناتباهیده است. بدین معنی که :
\begin{enumerate}
\item{
اگر
$S \in E[n]$
چنان باشد که برای هر
$T \in E[n]$،
$en(S,T)=1 $
آنگاه
$ S=\infty$.
}
\item{

اگر
$T \in E[n]$
چنان باشد که برای هر
$S \in E[n]$،
$en(S,T)=1 $
آنگاه
$ T=\infty$.
}
\end{enumerate}
}

\item{
برای هر
$S \in E[n]$:
$$ e_n(S,S)=1 $$
}

\item{
برای هر
$S,T \in E[n]$:
$$ e_n(T,S)=e_n(S,T)^{-1} $$
}

\item{
اگر
$\sigma : \bar{k} \rightarrow \bar{k}$
یک خودریختی باشد آنگاه برای هر
$S,T \in E[n]$:
$$ e_n(\sigma(S), \sigma(T)) = \sigma(e_n(S,T)) $$
}

\item{
اگر
$\varphi : E(\bar{k}) \rightarrow E(\bar{k})$
یک درون‌ریختی جداپذیر باشد، آنگاه برای هر
$S,T \in E[n]$:
$$ e_n(\varphi(S),\varphi(T)) = e_n(S,T)^{deg \varphi}$$
}
\end{enumerate}
\definition
نگاشت
$ e_n : E[n] \times E[n] \longrightarrow \mu_n $
تعریف شده در بالا را زوجیت وایل می‌گوییم.
\pagebreak
\section{همسانی }
فرض کنید 
$E_1 : {y_1}^2 = {x_1}^2+A_1x_1+B_1$
و
$E_2 : {y_2}^2 = {x_2}^2+A_2x_2+B_2$
دو خم تعریف شده روی میدان
$k$
با مشخصه‌ی مخالف ۲ و ۳ باشند، در این‌صورت یک همسانی از
$E_1$
به
$E_2$
یک همریختی به شکل
$\varphi : E_1(\bar{k}) \longrightarrow E_2(\bar{k})$
است که توسط توابع گویا تعریف شده باشد. بنابراین توابع گویای 
$R_1(x,y)$
و
$R_2(x,y)$
وجود دارند بطوریکه:
$$ \varphi : (R_1(x,y), R_2(x,y)) $$
یک همسانی را می‌توان به‌شکل 
$\varphi : (r_1(x),r_2(x)y)$
که
$r_1(x)$
و
$r_2(x)$
توابع گویا هستند نیز نشان داد.
\remark
$\varphi$
یک همریختی است یعنی:
$$ \forall P,Q \in E_1(\bar{X}) ~:~ \varphi(P+Q)=\varphi(P)+ \varphi(Q) $$
\remark
اگر
$E_1=E_2$
آنگاه، همسانی یک درون‌ریختی ناصفر خواهد بود.
\\
\definition
فرض کنید 
$\alpha = (r_1(x), r_2(x)y)$
و ضرایب 
$r_1$
و
$r_2$
در
$k$
قرار داشته باشند، گوییم 
$\alpha$
روی
$k$
تعریف شده است:
\begin{itemize}
\item{
اگر 
$r_1(x) = \frac{p(x)}{q(x)}$
، آنگاه:
$ deg(\alpha) = max \{ deg(p), deg(q) \} $
}

\item{
اگر 
${r_1}'(x) \ne 0$
آنگاه گفته می‌شود که
$\alpha$
جداپذیر است.
}
\end{itemize}~
\\
\theorem {
فرض کنید 
$\alpha : E_1(\bar{k}) \longrightarrow E_2(\bar{k})$
یک همسانی باشد، در این‌صورت:
\begin{enumerate}
\item{
اگر
$\alpha$
جداپذیر باشد:
$$ deg(\alpha) = \# ker(\alpha) $$
}
\item{
اگر
$\alpha$
جداناپذیر باشد:
$$ deg(\alpha) \geq \# ker(\alpha) $$
}

\end{enumerate} 
\remark
$ker(\alpha)$
 یا هسته همسانی یک زیرگروه متناهی از 
$E_1(\bar{k})$
 است.
}
\\
\section{دوگان همسانی}
\theorem 
فرض کنید 
$\alpha : E_1 \longrightarrow E_2$
یک همسانی باشد. در این‌صورت همسانی
$\hat{\alpha} : E_2 \longrightarrow E_1$
وجود دارد بطوریکه:
$$ \hat{\alpha} \circ  \alpha = [deg(\alpha)] $$
به 
$\hat{\alpha}$
دوگان همسانی
$\alpha$
گفته می‌شود. البته قابل ذکر است که
$\hat{\alpha}$
یکتاست و 
$$deg(\alpha) = deg(\hat{\alpha})$$
و همچنین :
$$ \alpha \circ \hat{\alpha} = [deg(\alpha)]$$
\\
\subsection{خواص دوگان همسانی}
اگر
$\varphi : E_1 \longrightarrow E_2$
و
$\psi : E_2 \longrightarrow E_3$
دو همسانی باشند، آنگاه :
\begin{enumerate}[label=\alph*.]
\item{
$ \widehat{\varphi \circ \psi} = \hat{\varphi} \circ \hat{\psi}$
}

\item{
$ \hat{ \hat{ \varphi } } = \varphi$
}
\item{
$ deg(\varphi \circ \hat{ \varphi }) = (deg \varphi)^2 $
}
\end{enumerate}~
\\
\theorem 
فرض کنید
$E$
یک خم بیضوی و 
$C$
زیرگروهی متناهی از 
$E$
باشد. یک خم بیضوی یکتای
$E'$
و یک همسانی جداپذیر
$\varphi : E \longrightarrow E' \cong \frac{E(\bar{k})}{ker(\varphi)}$
 وجود دارد بطوریکه:
$$ ker(\varphi) = C $$

\proposition
اگر
$\ell$
یک عدد اول متباین با مشخصه‌ی میدان باشد آنگاه 
$E[\ell]$
دارای 
$\ell + 1$
زیرگروه از مرتبه‌ی
$\ell$
خواهد بود. هر یک از این زیرگروه‌ها با توجه به قضیه‌ی قبل می‌تواند هسته‌ی یک همسانی باشند:
$$ E[\ell] = \{ P \in E(\bar{K}) | \ell P = \infty \} $$
\refproof
می‌دانیم 
$E[\ell] \cong \mathbb{Z}_{\ell} \times \mathbb{Z}_{\ell}$
، بنابراین 
${\ell}^2$
عضو دارد. از آنجایی که
$\mathbb{Z}_{\ell} \times \mathbb{Z}_{\ell}$
دوری نیست، همه‌ی عناصر غیربدیهی 
$E[\ell]$
از مرتبه‌ی 
$\ell$
هستند( زیرا تنها 
$\ell$
،
${\ell}^2$
را می‌شمارد)، بنابراین با حذف عنصر بدیهی
$E[\ell]$
، 
${\ell}^2-1$
عنصر باقی می‌ماند. هر زیرگروه
$E[\ell]$
شامل عنصر بدیهی است، عنصر بدیهی را از هر گروه حذف می‌کنیم. هر زیرگروه 
$\ell -1$
عنصر غیربدیعی خواهد داشت، از آنجایی که
$\frac{{\ell}^2-1}{{\ell}-1} = \ell + 1$
، پس
$E[\ell]$
،
${\ell}+1$
زیرگروه از مرتبه‌ی
$\ell$
خواهد داشت.
\\
\proposition
اگر
$\ell$
یک عدد اول متباین با مشخصه‌ی میدان باشد، هر همسانی که هسته‌ی آن زیرگروهی از
$E[\ell]$
باشد از درجه‌ی 
$\ell$
خواهد بود و به آن یک 
$\ell$-همسانی
می‌گوییم.
\\
\theorem
اگر
$\phi : E_1 \longrightarrow E_2$
یک همسانی باشد آنگاه
$E_1$
سوپرسینگولار است اگر وتنها اگر
$E_2$
سوپرسینگولار باشد ( و بالعکس).
\\
\theorem[تیت]
دو خم بیضوی
$E_1$
و
$E_2$
روی میدان متناهی
$\mathbb{F}_q$
همسان هستند اگر و تنها اگر
$$ \# E_1(\mathbb{F}_q) = \# E_2(\mathbb{F}_q) $$
به‌عبارت دیگر دو خم بیضوی
$E_1$
و
$E_2$
را همسان گوییم هرگاه یک همسانی از
$E_1$
به
$E_2$
وجود داشته باشد.
\corollary
با استفاده از قضیه‌ی تیت می‌توان در زمان چندجمله‌ای مشخص کرد که آیا دو خم روی میدان متناهی
$\mathbb{F}_q$
، همسان هستند یا خیر.
\\
\subsection{فرمول ولو}
\theorem
فرض کنید 
$E$
یک خم بیضوی با معادله‌ی وایرشتراس یر روی میدان
$k$
باشد:
$$ y^2+a_1xy+a_3y = x^3+a_2x^2+a_4x+a_6 $$
و
$C$
یک زیرگروه متناهی از
$E(\bar{k})$
باشد. در اینصورت یک خم بیضوی
$E'$
و یک همسانی جداپذیر 
$\alpha$
از
$E$
به
$E'$
وجود دارد بطوریکه:
$$C=ker(\alpha)$$
با استفاده از مراحل زیر می‌توان خم
$E'$
و همسانی
$\alpha$
را به‌دست آورد:
\begin{enumerate}
\item{
قرار می‌دهیم:
$$ F(x,y) = x^3+a_2x^2+a_4x+a_6 - y^2+a_1xy+a_3y $$
و برای هر
$Q=(x_a,y_a) \in c$
که
$Q \ne \infty$
تعریف می‌کنیم:
$$ g_Q^x = F_x(Q) = 3x_Q^2+2a_2x_Q+a_4-a_1y_Q$$
$$ g_Q^x = F_y(Q) = -2y_Q-a_1x_Q-a_3 $$
\\
\begin{equation*}
V_Q = 
\begin{cases}
& g_Q^x ~~, 2Q=\infty ~ \text{اگر} \\
& 2g_Q^x - a_1g_Q^y ~~,2Q \ne \infty ~ \text{اگر} 
\end{cases}
\end{equation*}
\\
$$ u_Q = {\Big( g_Q^y \Big)}^2 $$
}
\item{
فرض کنید
$C_2$
نقاطی از مرتبه‌ی ۲ در
$C$
باشند، 
$R \subseteq C$
را طوری انتخاب می‌کنیم که یک اجتماع مجزا به شکل زیر داشته باشیم:
$$ C = {\infty} \cup C_2 \cup R \cup (-R) $$
به‌عبارت دیگر هر نقطه‌ی
$p,-p \in C$
که تابی از مرتبه‌ی ۲ نباشند را درنظر می‌گیریم، دقیقا یکی از آنها را در
$R$
قرار می‌دهیم.
\\
حال فرض کنید
$S=R \cup C_2$
قرار می‌دهیم:
$$ v = \sum\limits^{}_{Q \in S} (v_Q) ~~,~~  w = \sum\limits^{}_{Q \in S}(u_Q+x_Qv_Q) $$
در این‌صورت خم
$E'$
دارای معادله‌ی :
$$Y^2+A_1XY+A_3Y = X^3+A_2X^2+A_4X+A_6$$
است که در آن:
$$ A_1=a_1 ~,~ A_2=a_2 ~,~ A_3=a_3 ~,~ A_4=a_4-5v ~,~ A_6=a_6-(a_1^2+4a_2)v-7w $$
}

\item{
همسانی
$\alpha : E_{(x,y)} \longrightarrow E'_{(X,Y)}$
که در آن:
$$ X = x + \sum\limits^{}_{Q \in S} \big(\frac{v_Q}{x-x_Q} + \frac{u_Q}{(x-x_Q)^2} \big) $$
$$
 Y = y - \sum\limits^{}_{Q \in S} \big( v_Q \cdot \frac{2y+a_1x+a_3}{(x-x_Q)^3} + v_Q \cdot \frac{a_1(x-x_Q)+y-y_Q}{(x-x_Q)^2} + 
\frac{a_1u_Q - g^{x}_{Q} g^y_Q}{(x-x_Q)^2} \big) 
$$
را تعریف می‌کنیم. در واقع
$$ \alpha : E \longrightarrow E' \cong p \rightsquigarrow \big(X(p),Y(p) \big) $$
که در آن :
$$ X(p) = x(p) + \sum\limits^{}_{Q \in C} [~x(P+Q) - x(Q)~] $$
$$ Y(p) = y(p) + \sum\limits^{}_{Q \in C} [~y(P+Q) - y(Q)~] $$
}
\end{enumerate}~
\\
\example
فرض کنید 
$E:x^3+ax^2+bx$
یک خم بیضوی تعریف شده روی میدان 
$k$
باشد. نقطه‌ی 
$(0,0)$
یک نقطه‌ای از مرتبه‌ی ۲ روی این خم است. بنابراین
$C=\{ \infty , (0,0) \}$
زیرگروهی از
$E(\bar{k})$
است. می‌خواهیم یک همسانی از
$E$
به یک خم بیضوی
$E'$
تعریف کنیم که هسته‌ی آن
$C$
باشد. بنابراین داریم:
$$ C_2 = \{ (0,0) \} ~,~ R = \phi ~,~ S={(0,0)} $$
$$ F(x,y) = x^3+ax^2+bx-y^2 $$
$$ g^x_Q = F_x(Q) = 3x^2_Q + 2ax_Q + b = b $$
$$ g^y_Q  = F_y(Q) = -2y_Q = 0$$
$$ v_Q = g^x_Q = b ~~,~~ u_Q=(g^y_Q)^2 = 0 $$
$$ v = \sum\limits^{}_{Q \in S} v_Q = b ~~,~~ w = \sum\limits^{}_{Q \in S}(u_Q + x_Qv_Q) = 0$$
$$ Y^2+A_1XY+A_3Y = X^3+A_ 2X^2+A_4X+A_6$$
$$ A_1=a_1=0 ~~,~~ A_2=a_2=a ~~,~~ A_3=a_3=0 ~~,~~ A_4=a_4-5v = b-5b = -4b ~~,~~ A_6 = a_6-(a^2_1 + 4a_2)v-7w = -4b^2$$
$$ \Longrightarrow E': Y^2=X^3+a_2X^2-4bY-4ab $$
$$  $$
\begin{equation*}
\alpha : E \longrightarrow E' \cong (x,y) \longrightarrow (X,Y) \Rightarrow
\begin{cases}
X = x + \frac{b}{x} \\
Y = y - \frac{by}{x^2}
\end{cases}
\end{equation*}
